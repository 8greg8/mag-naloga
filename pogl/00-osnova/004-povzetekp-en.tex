\abstractp
Measurement devices for indirect calorimetry, which are very expensive devices with a mask that must be attached on the head, are not suitable for widespread use. For the needs of field investigations, non-calarimetric methods have been developed that focus on kinematic observation. These include contact and contactless methods. Contact methods are relatively inaccurate, and they also limit movement, thus indirectly affect the result. In non-contact methods, methods for analyzing video sequences andthe construction of metabolic models are prevalent. These are complicated and time consuming. They are also limited to specific movements. None of the contactless methods of estimating energy consumption does not exploit the motion field and its approximations, which best describes the movement.

In this paper we explored novel contactless method for physiological parameters estimation from motion. For determining physiological parameters we used optical and scene flow algorithms combined with HOOF and HAFA descriptors for robustness. SVM regression with developed \nurbf grid search optimisation was used to predict energy expenditure and heart rate of the observed subject. Tracker, Kalman and Gaussian filter were used to get rid of background noise and movement of non-observable objects.

Results show that selected physiological parameters are \emph{observable}. Proposed method can be used with different camera systems from different viewpoints. Better results can be achieved with viewpoint mixing. Initial methods from phase 1 experiments are not suitable for field testing. When training, fatigue accumulation must be taken into account. For laboratory experiments it is better to use optical flow method. In field best results can be achieved with scene flow usage.

\keywords physical activity, energy expenditure, heart rate, optical flow, scene flow, support vector machine, RBF kernel, KCF tracker, Kinect sensor, squash