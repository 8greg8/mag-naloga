\seznamsimbolov

V zaključnem delu so uporabljeni naslednje veličine in simboli:

\begin{table}[h]
\centering
%\begin{footnotesize}
\begin{tabular}{l l l l}
 \toprule
 \multicolumn{2}{c}{\bf{Veličina / oznaka}} & \multicolumn{2}{c}{\bf{Enota}}  \\
 \midrule
Ime & Simbol & Ime & Simbol \\
 \midrule
 \multirow{2}{*}{energija} & \multirow{2}{*}{$W$} & kilojoul & \si{\kjoul} \\
 && kilokalorija & \si{\kcal} \\
 energijska poraba & $W_p$ &  & \si{\kcal.\min^{-1}} \\
 masa & $m$ & kilogram & kg \\
 srčni utrip & $hr$ & udarci na minuto & bpm \\
 srčni utrip v mirovanju & $hr_r$ & udarci na minuto & bpm \\
 teoretični maksimalni srčni utrip & $hr_{tmax}$ & udarci na minuto & bpm \\
 poraba kisika & ${VO}_{2}$ &  & \si{\ml.\kg^{-1}.\min^{-1}} \\
 višina & $h$ & centimeter & cm \\
 korelacijski koeficient & $R$ & odstotek & \% \\
 spol & $s$ &  & \\
 starost & $st$ & leto & \\
 nasičenost kisika & $SpO_2$ & odstotek & \% \\
 
  \bottomrule
\end{tabular}
%\end{footnotesize}
  \caption{Veličine in simboli}
  \label{prebojne_trdnosti}
\end{table}

Pri čemer so vektorji in matrike napisani s poudarjeno pisavo.
Natančnejši pomen simbolov in njihovih indeksov je razviden iz
ustreznih slik ali pa je pojasnjen v spremljajočem besedilu, kjer je
simbol uporabljen.