\povzetekp
Merjenje porabe energije je pomembno v športni znanosti in medicini, še posebej kadar želimo oceniti obseg in intenzivnost fizične aktivnosti. Večinoma so pristopi še vedno odvisni od senzorjev ali markerjev, ki jih nameščamo neposredno na telo. V tem delu predstavljamo nov pristop, ki uporablja popolnoma brezkontaktno, avtomatsko metodo, ki temelji na uporabi algoritmov računalniškega vida in cenenih, široko dostopnih slikovnih senzorjev. Pri tem se zanašamo na oceno optičnega in prostorskega toka za izračun histogramov orientiranega optičnega toka (HOOF), ki smo jih dopolnili s histogrami absolutnih tokovnih amplitud (HAFA). Deskriptorje uporabljamo v regresijskem modelu, ki nam omogoča, da ocenimo porabo energije in v manjši meri srčni utrip. Naša metoda je bila testirana v laboratorijskem okolju in v realnih pogojih športne tekme. Rezultati potrjujejo, da bi lahko energijsko porabo dobili iz izključno brezkontaktnega opazovanja. Predlagano metodo lahko uporabljamo na različne načine, vključno z bližnjo infrardečo kamero, ki razširja prihodnji potencial.


\kljucnebesede intenzivnost fizične aktivnosti, energijska poraba, srčni utrip, optični tok, prostorski tok, strojno učenje z metodo podpornih vektorjev, RBF jedro, KCF sledilnik, Kinect senzor, squash