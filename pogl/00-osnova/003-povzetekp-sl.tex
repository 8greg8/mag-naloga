\povzetekp
Merjenje porabe energije je pomembno v športni znanosti in medicini, še posebej kadar želimo oceniti obseg in intenzivnost fizične aktivnosti. Večinoma so pristopi še vedno odvisni od senzorjev ali markerjev, ki jih nameščamo neposredno na telo. V tem delu predstavljamo nov pristop, ki uporablja popolnoma brezkontaktno, avtomatsko metodo, ki temelji na uporabi algoritmov računalniškega vida in cenenih, široko dostopnih slikovnih senzorjev. Pri tem se zanašamo na oceno optičnega in prostorskega toka za izračun histogramov orientiranega optičnega toka (HOOF), ki smo jih dopolnili s histogrami absolutnih tokovnih amplitud (HAFA). Deskriptorje uporabljamo v regresijskem modelu, ki nam omogoča, da ocenimo porabo energije in v manjši meri srčni utrip. Naša metoda je bila testirana v laboratorijskem okolju in v realnih pogojih športne tekme. Podlaga tega dela je obsežna študija, kjer smo preizkusili različne modalitete vizualnih podatkov (barvne in infrardeče kamere ter kamere na podlagi čas preleta), različne tipe senzorjev, ter različne kombinacije algoritmov v procesnem cevovodu, ki obsega sledenje, modeliranje, napovedovanje in filtriranje rezultatov.
Rezultati potrjujejo, da bi lahko energijsko porabo merili izključno na podlagi takšnega brezkontaktnega opazovanja. Majhen del rezultatov naše študije je že bil objavljen na mednarodni konferenci iz področja računalniškega vida, večina rezultatov pa bo poslana v objavo v obliki članka v primerni znanstveni reviji.

\kljucnebesede fizična aktivnost, energijska poraba, srčni utrip, optični tok, prostorski tok, strojno učenje, RBF jedro, KCF sledilnik, Kinect senzor, squash