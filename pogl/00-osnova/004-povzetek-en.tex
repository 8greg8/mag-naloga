\abstract
Measurement of energy expenditure is an important tool in sport science and medicine, especially when trying to estimate the extent and intensity of physical activity. However, most approaches still rely on sensors or markers, placed directly on the body. In this work, we present a novel approach, using a fully contactless, automatic method, that relies on computer vision algorithms and widely available and inexpensive imaging sensors. We rely on the estimation of the optical and scene flow to calculate Histograms of Oriented Optical Flow (HOOF) descriptors, which we subsequently augment with the Histograms of Absolute Flow Amplitude (HAFA). Descriptors are fed into regression model, which allows us to estimate energy consumption, and by lesser extent, the heart rate. Our method has been tested both in lab environment and in realistic conditions of a sport match. This work is based on a comprehensive study, where we tested different modalities of visual data (color and infrared cameras, time-of-flight cameras), different sensor types, and different combinations of algorithms in the processing pipeline, which consists of tracking, modeling, predicting and filtering of the results. Results confirm that energy expenditure could be derived from purely contactless observations using our approach. Small subset of our study has already been published at the international computer vision conference, however the rest of the results will be submitted to the top-level scientific journal.

\keywords physical activity, energy expenditure, heart rate, optical flow, scene flow, machine learning, RBF kernel, KCF tracker, Kinect sensor, squash