\section{Model gibanja}\label{sec:model-gibanja}
% Ponovi teorijo energijske porabe in kako pridemo do gibanja
% Dokazi, da lahko gibanje opazujemo s kamerami
% Teorija, da je najboljši približek optični tok
Predpostavljamo, da poraba energije v človeškem telesu zaradi gibanja telesa prevladuje nad vsemi
drugimi vzroki. Zato se ta komponenta porabe energije lahko izpelje z opazovanjem kinematike~\cite{levine2005measurement} človeškega telesa, če se vsi drugi vzroki za porabo energije štejejo za šum.

Predpostavimo perspektivni model kamere, ki je opisan v poglavju~\ref{sec:model-kamere}.
Naj bo delec z maso $m$ predstavljen kot točka prizora $\vec{p}$, na sliki~\ref{fig:model-gibanja}. Gibanje delca $\vec{p}$ lahko predstavimo z vektorjem hitrosti $\vec{v} = [v_X~v_Y~v_Z]^\top$, $\vec{v} \in \mathcal{V} \subset \mathbb{R}^3$, kjer so $v_X$, $v_Y$ in $v_Z$ hitrosti glede na osi in $\mathcal{V}$ vektorski prostor. Kadar imamo v prostoru več masnih delcev, množico vektorjev hitrosti imenujemo \emph{polje hitrosti} (angl. Velocity Field) $\mathbf{H}: \mathbb{R}^3 \to \mathcal{V}$, kjer velja $\vec{p} \mapsto \vec{v}$~\cite{trucco1998introductory}.


\begin{figure}[htb]
\centering
\begin{tikzpicture}%[tdplot_main_coords, scale=0.5]
[x={(0.8cm,0.4cm)}, y={(0cm,1cm)}, z={(0.8cm,-0.4cm)}, scale=0.5]
      
	% Coordinate system
    \coordinate (O) at (0,0,0);
    \coordinate (y) at (0,5,0);
    \coordinate (z) at (0,0,15);
    \coordinate (x) at (10,0,0);
    \draw [axis] (O) -- (y) node [above] {$y$};
    \draw [axis] (O) -- (z) node [below] {$z$};
    \draw [axis] (O) -- (x) node [below] {$x$};
    \node (izhodisce) [below] at (O) {$o$};
    
    % delec
    \coordinate (p) at (5,3,10);
    \draw [fill=black] (p) circle (1.5mm) node [below] {$\vec{p}$};
    \draw [dash] (O) -- (p);
    \draw [dash] (5,0,0) node [above] {$X$} -- (5,0,10);
    \draw [dash] (0,0,10) node [below] {$Z$} -- (5,0,10);
    \draw [dash] (5,0,10) -- (5,3,10);
    
    % hitrost
    \draw [velocity] (p) -- (6,4,10) node [above] {$\vec{v}$};
    

\end{tikzpicture}
\caption[Predstavitev delca $\vec{p}$ v koordinatnem sistemu kamere]{Predstavitev delca $\vec{p}$ v koordinatnem sistemu kamere. Delec je predstavljen kot točka prizora z vektorjem hitrosti $\vec{v}$~\cite{trucco1998introductory}.}
\label{fig:model-gibanja}
\end{figure}



Relativno gibanje delca $\vec{p}$ glede na koordinatno izhodišče kamere $\vec{O}$ lahko opišemo kot:

\begin{equation}
	\vec{v} = -\vec{T}-\vec{\omega}\times\vec{p},
\end{equation}

kjer je $\vec{T}$ translatorna hitrost in $\omega$ kotna hitrost~\cite{trucco1998introductory}. Po komponentah lahko gibanje opišemo z enačbo~\eqref{eq:gibanje}

\begin{equation} \label{eq:gibanje}
	\begin{bmatrix}
	v_X \\ v_Y \\ v_Z
	\end{bmatrix}
    =
    \begin{bmatrix}
    - T_X - \omega_Y Z + \omega_Z Y \\
    - T_Y - \omega_Z X + \omega_X Z \\
    - T_Z - \omega_X Y + \omega_Y X
    \end{bmatrix}.
\end{equation}

Gibanje telesa v prostoru torej lahko opišemo s poljem hitrosti $\vec{H}$.


