\section{Optični tok} \label{sec:opticni-tok}
% Teorija optičnega toka
% vrste optičnega toka
% Kateri tip smo mi uporabili
% Zakaj smo tega uporabili
% Kako smo ga uporabili
% Problemi naše predpostavke
% Rešitev s kalibracijo velikosti
Za namen razlage upoštevamo perpektivni model kamere iz poglavja \ref{sec:model-kamere} in model gibanja iz poglavja \ref{sec:model-gibanja}. Dodatno upoštevamo da se osvetlitev ne spreminja.

$\vec{q}$ je slika delca $\vec{p}$ na slikovni ravnini $\mathit{\Omega}$ Delec in njegova slika sta predstavljena na sliki \ref{fig:optical-flow} S časovnim odvodom enačbe \eqref{eq:slika-delca}, dobimo hitrost delca na slikovni ravnini:

\begin{equation}\label{eq:hitrost-slike-delca}
	\vec{u} = f \frac{Z\vec{v}-v_Z\vec{p}}{Z^2},
\end{equation}

kjer je $\vec{u} \in \mathcal{U} \subset \mathbb{R}^2$.




\begin{figure}[htb]
\centering
\begin{tikzpicture}%[tdplot_main_coords, scale=0.5]
[x={(0.8cm,0.4cm)}, y={(0cm,1cm)}, z={(0.8cm,-0.4cm)}, scale=0.5]

	% Coordinate system
    \coordinate (O) at (0,0,0);
    \coordinate (oo) at (0,0,5);
    \coordinate (y) at (0,5,0);
    \coordinate (z) at (0,0,15);
    \coordinate (x) at (10,0,0);
    \draw [axis] (O) -- (y) node [above] {$y$};
    \draw [base-axis] (O) -- (oo);
    \draw [axis] (O) -- (x) node [below] {$x$};
    \node (izhodisce) [below] at (O) {$o$};
    
    % Image plane
    \coordinate (ol) at (-5,0,5);
    \coordinate (or) at (5,0,5);
    \coordinate (ot) at (0,3,5);
    \coordinate (ob) at (0,-3,5);
    \coordinate (lb) at (-5,-3,5);
    \coordinate (rb) at (5,-3,5);
    \coordinate (lt) at (-5,3,5);
    \coordinate (rt) at (5,3,5);
    \draw [plane] (lb) -- (lt) -- (rt) -- (rb) -- cycle;
    \draw [dash] (ol) -- (or);
    \draw [dash] (ob) -- (ot);
    \node [xshift=3mm, yshift=5mm] at (lb) {$\mathit{\Omega}$};
    % Draw the rest of axis
    \draw [axis] (oo) -- (z) node [below] {$z$};
    
    % focal length
    \coordinate (of) at (-5,0,0);
   	\draw [dash] (O) -- (of);
    \draw [<->] (of) -- (ol) node [below] at (-5,0,2.5) {$f$};
    
    % delec
    \coordinate (p) at (5,3,10);
    \draw [fill=black] (p) circle (1.5mm) node [below] {$\vec{p}$};
    \draw [dash, name path=line 1] (O) -- (p);
    \draw [dash] (5,0,0) node [above] {$X$} -- (5,0,10);
    \draw [dash] (0,0,10) node [below] {$Z$} -- (5,0,10);
    \draw [dash] (5,0,10) -- (p);
    
    % hitrost
    \coordinate (v) at (6,4,10);
    \draw [velocity] (p) -- (v) node [above] {$\vec{v}$};
    \draw [dash] (O) -- (v);
    
    % slika delca
    \coordinate (q) at (2.5,1.5,5);
    \draw [fill=black] (q) circle (1mm) node [below] {$\vec{q}$};
    \draw [dash] (2.5,0,5) node [below] {$x$} -- (q);
    \draw [dash] (0,1.5,5) node [left] {$y$} -- (q);
    
    %hitrost
    \draw [velocity] (q) -- (3,2,5) node [above] {$\vec{u}$};

\end{tikzpicture}
\caption[Preslikava hitrosti delca na slikovno ravnino $\mathit{\Omega}$]{ Preslikava hitrosti delca na slikovno ravnino $\mathit{\Omega}$. Gibajoči delec $\vec{p}$ ima sliko $\vec{q}$. Hitrost delca $\vec{v}$ ima sliko hitrosti $\vec{u}$, ki predstavlja idealni vektor gibanja. Koordinatni sistem predstavlja sistem kamere.}
\label{fig:optical-flow}
\end{figure}




Razširjena oblika enačbe \eqref{eq:hitrost-slike-delca}, kjer upoštevamo \eqref{eq:gibanje}, je zapisana z enačbo \eqref{eq:hitrost-slike-delca-raz} \cite{trucco1998introductory}. Prvi člen v posamezni enačbi predstavlja \textbf{translatorni del}, ostali členi pa sodijo v \textbf{rotacijski del}.

\begin{equation}\label{eq:hitrost-slike-delca-raz}
\begin{aligned}
	u_x = & \frac{T_Z x - T_X f}{Z} - \omega_Y f + \omega_Z y + \frac{\omega_X x y}{f} - \frac{\omega_Y x^2}{f} \\
    u_y = & \frac{T_Z y - T_Y f}{Z} - \omega_X f + \omega_Z x + \frac{\omega_Y x y}{f} - \frac{\omega_X y^2}{f}
\end{aligned}
\end{equation}

Kadar imamo na slikovni ravnini več slik delcev, množico vektorjev hitrosti $\vec{u}$ imenujemo \textbf{polje gibanja} (angl. Motion Field) $\vec{G} : \mathit{\Omega} \to \mathcal{U}$, kjer velja $ \vec{q} \mapsto \vec{u}$ \cite{trucco1998introductory}. Polje gibanja $\vec{G}$ lahko razumemo kot projekcijo polja hitrosti $\vec{H}$ na slikovno ravnino, zato ta predstavlja idealno rekonstrukcijo gibanja. V praksi do polja gibanja ne moremo dostopati, zato se poslužujemo njegovih približkov.  

Video posnetek je sestavljen iz sekvence slik, to pa lahko opišemo kot funkcijo osvetljenosti slikovnega elementa $I(\vec{x},t)$, na poziciji $\vec{x} = [x~y]^\top$ ob času $t$ \cite{wedel2011stereo}. Gibanje oseb opazimo kot premikanje pikslov skozi čas, pri čemer predpostavimo, da osvetljenost posameznega piklsa ostaja konstantnta \cite{trucco1998introductory}. Stacionarnost osvetljenosti slikovnega elementa lahko opišemo z enačbo  

\begin{equation}
	\frac{d I(\vec{x}, t)}{dt} = \frac{\partial I}{\partial x} \frac{dx}{dt} + \frac{\partial I}{\partial y} \frac{dy}{dt} + \frac{\partial I}{\partial t} = 0,
\end{equation}

to pa lahko zapišemo z vektorjem hitrosti slikovnega elementa $\vec{w} \in \mathcal{W} \subset \mathbb{R}^2$ v kompaktnejšo obliko

\begin{equation}\label{eq:opticni-tok}
	(\nabla I)^\top \vec{w} + I_t = 0.
\end{equation}

Enačba \eqref{eq:opticni-tok} predstavlja \textbf{omejitev optičnega toka} \cite{trucco1998introductory}. Če v enačbi \eqref{eq:opticni-tok} normaliziramo prostorski gradient $(\nabla I)$, v enačbi \eqref{eq:aperture-problem} opazimo, da lahko  določimo le hitrost, ki je vzporedna prostorskemu gradientu. Pojav je znan kot problem reže (angl. Aperture problem) \cite{trucco1998introductory}. 

\begin{equation}\label{eq:aperture-problem}
	\frac{(\nabla I)^\top \vec{w}}{\| \nabla I \|} = - \frac{I_t}{\| \nabla I \|} = w_n
\end{equation}

\textbf{Problem reže} si lahko razlagamo na način opazovanja gibanja daljice na beli podlagi skozi režo tako, da ne vidimo koncev. Zaradi omejene vizualne informacije lahko določimo hitrost le v pravokotni smeri na daljico \cite{trucco1998introductory}. Razlaga je predstavljena na sliki \ref{fig:aperture-problem}.




\begin{figure}[htb]
\centering
\begin{tikzpicture}[scale=0.7]
\tikzset{aperture/.style = {
fill=teal!50, 
draw=teal!50!black!80
}}
\tikzset{stick/.style = {
fill=orange!50!black!50, 
draw=orange!50!black!80, solid, thick,
minimum width = 1mm, minimum height=7cm
}}
  \begin{scope}
  		\coordinate (top) at (10,10);
        \coordinate (bottom) at (0,0);
       	\coordinate (center) at (5,5);

  		% palica
        \coordinate (pc) at (4,4);
        \path (pc) node[stick,rotate = 45]{};
        \draw [fill] (pc) circle (0.5mm);
        \coordinate (pc2) at (6,4);
        \path (pc2) node[stick, rotate = 45, dashed, fill=none]{};
        \draw [fill] (pc2) circle (0.5mm);
        
        % vektor
        \draw [velocity, draw=black] (pc) -- ++(1,1) node [below] {${\vec{w}_n}$};
        \draw [velocity, draw=black] (pc) -- (pc2) node [below] {$\vec{w}$};
  		% aperture	
		\draw [aperture, opacity=0.8] (bottom) rectangle (top) (center) circle (20mm);  
  \end{scope}


\end{tikzpicture}
\caption[Problem reže]{Problem reže. Ker skozi režo ne vidim koncev daljice, lahko določimo le hitrost v pravokotni smeri na daljico \cite{trucco1998introductory}.}
\label{fig:aperture-problem}
\end{figure}




Kadar imamo na slikovni ravnini več premikajočih slikovnih elementov, vektorsko polje hitrosti $\vec{w}$ imenujemo \textbf{optični tok} (angl. Optical flow) $\vec{O}: \mathit{\Omega} \to \mathcal{W}$, kjer velja $ \vec{q} \mapsto \vec{w}$ \cite{trucco1998introductory}. Optični tok je dobra aproksimacija polja gibanja v točkah visokega prostorskega gradienta svetlosti in konstantne osvetlitve.



\subsection{Metode estimacije optičnega toka}\label{sec:metode-of}

Metode estimacije optičnega toka $\vec{O}$ v grobem delimo na diferencialne in {ujemalne} metode \cite{trucco1998introductory}. Z \textbf{diferencialnimi metodami} računamo optični tok z uporabo parcialnih diferencialnih enačb ali minimizacijskimi metodami. Z metodami dobimo \textbf{gost optični tok}, kar pomeni, da je optični tok določen za vsak slikovni element \cite{trucco1998introductory}. Te metode zelo natačno opisujejo optični tok in ne proizvajajo vrednosti, ki lokalno odstopajo, zato je optični tok gladek \cite{brox2011large}.  Glavni problem teh metod je, da so računsko zelo zahtevne \cite{trucco1998introductory}.

Pri \textbf{ujemalnih metodah} računamo optični tok le na značilnih točkah \cite{trucco1998introductory}. Zaradi uporabe značilk so te metode lahko bolj efektivne, saj ne potrebujemo določevanja korespondenc za vse piksle. Prav tako se lahko uporabijo za računanje optičnega toka v realnem času, saj niso računsko zahtevne. Po \cite{trucco1998introductory} je največja težava teh metod, da računajo \textbf{redek optični tok}, saj je ta določen le za slikovne elemete, ki predstavljajo značilne točke. Prav tako delujejo dobro le pri majhnih premikih, ker temeljijo na Taylorjevi aproksimaciji enačbe \eqref{eq:opticni-tok} \cite{wedel2011stereo}. 

V nadaljevanju predstavljamo diferencialno metodo Farneb{\"a}ck algoritem.

\subsubsection{Farneb{\"a}ck algoritem.}
Alogritem temelji na estimaciji premika z razčlenjevanjem polinoma  po enačbi \eqref{eq:polinom}, kjer je $\vec{A}$ simetrična matrika, $\vec{b}$ vektor in $c$ skalar \cite{farneback2003two}.

\begin{equation}\label{eq:polinom}
	f(\vec{x}) \sim \vec{x}^\top \vec{A} \vec{x} + \vec{b}^\top \vec{x} + c
\end{equation}

Ideja temelji na tem, da aproksimiramo okolico piksla s kvadratičnim polinomom, pri čemer želimo najti premik piksla na poziciji $\vec{x}$ z minimizacijo enačbe \eqref{eq:polinom-min} in omejitvijo \eqref{eq:omejitev-polinoma}. $\vec{A}_1(\vec{x})$ in $\vec{b}_1(\vec{x})$ sta razčlenitvena koeficienta za prvo sliko, $\vec{A}_2(\vec{x})$ in $\vec{b}_2(\vec{x})$ koeficienta za drugo sliko in $w(\Delta\vec{x})$ je utežna funkcija za sosedne točke.

\begin{align}
\vec{A}(\vec{x}) = & \frac{\vec{A}_1(\vec{x} + \vec{A}_2(\vec{x}))}{2} \\
\Delta\vec{b}(\vec{x}) = & - \frac{1}{2}\left(\vec{b}_2(\vec{x}) - \vec{b}_1(\vec{x})\right) 
\end{align}

\begin{equation}\label{eq:polinom-min}
\sum_{\Delta x \in I} w(\Delta\vec{x}) \| \vec{A}(\vec{x} + \Delta\vec{x})\vec{d}(\vec{x}) - \Delta\vec{b}(\vec{x} +\Delta\vec{x}) \|^2
\end{equation}

\begin{equation}\label{eq:omejitev-polinoma}
\vec{A}(\vec{x})\vec{d}(\vec{x}) = \Delta\vec{b}(\vec{x})
\end{equation}

Rešitev minimizacije enačbe \eqref{eq:polinom-min} je enačba \eqref{eq:polinom-resitev}

\begin{equation}\label{eq:polinom-resitev}
 \vec{d}(\vec{x}) = \left( \sum w \vec{A}^\top \vec{a} \right)^{-1} \sum w \vec{A}^\top \Delta\vec{b}
\end{equation}

Evaluacija algoritma je bila narejena v \cite{Geiger2012CVPR}. Rezultati so povzeti v tabeli \ref{tab:farneback}. Algoritem so preverjali s procesorjem z 1 jedrom \@ \SI{2.5}{GHz}.

\begin{table}[htb]
	\centering
    \begin{tabular}{S[table-format=2.2] S[table-format=2.2] S[table-format=2.1] S[table-format=2.1] S[table-format=3.2] S[table-format=1]}
    \toprule
    \multicolumn{1}{c}{\textbf{Out-Noc}} & \multicolumn{1}{c}{\textbf{Out-All}} & \multicolumn{1}{c}{\textbf{Avg-Noc}} & \multicolumn{1}{c}{\textbf{Avg-All}} & \multicolumn{1}{c}{\textbf{Gostota}} & \multicolumn{1}{c}{\textbf{Čas izvajanja}} \\
    \midrule
    47.59~\% & 54.00~\% & 17.3~px & 25.3~px & 100.00~\% & 1~s\\
    \bottomrule
    \end{tabular}
    \caption[Evaluacija Farneb{\"a}ck algoritma v KITTI Vision Benchmark 2012]{Evaluacija Farneb{\"a}ck algoritma v KITTI Vision Benchmark 2012 \cite{Geiger2012CVPR}. Metrika Out-Noc predstavlja procent pikslov, ki težijo k napakam v območju, kjer ni prekrivnosti. Out-all je procent pikslov, ki težijo k napakam v celoti. Avg-Noc je povprečna napaka disparitete v območjih neprekrivnsoti. Avg-All je povprečna napaka disparitete v celoti. Gostota predstavlja procent pikslov, za katere je metoda določila referenco \cite{Geiger2012CVPR}.}
    \label{tab:farneback}
\end{table}
