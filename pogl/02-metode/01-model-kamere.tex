\section{Geometrijski model kamere}\label{sec:model-kamere}
Za geometrijski model kamere uporabimo perspektivni model, ki je predstavljen s sliko \ref{fig:perspektivni-model}. Sestavljen je iz slikovne ravnine, koordinatnega sistema kamere in točke prizora \cite{trucco1998introductory}.


\begin{figure}[htb]
\centering
\begin{tikzpicture}%[tdplot_main_coords, scale=0.5]
[x={(0.8cm,0.4cm)}, y={(0cm,1cm)}, z={(0.8cm,-0.4cm)}, scale=0.5]

	% Coordinate system
    \coordinate (O) at (0,0,0);
    \coordinate (oo) at (0,0,5);
    \coordinate (y) at (0,5,0);
    \coordinate (z) at (0,0,15);
    \coordinate (x) at (10,0,0);
    \draw [axis] (O) -- (y) node [right] {$y$};
    \draw [base-axis] (O) -- (oo);
    \draw [axis] (O) -- (x) node [below] {$x$};
    \node (izhodisce) [below] at (O) {$\vec{O}$};
    
    % Image plane
    \coordinate (ol) at (-5,0,5);
    \coordinate (or) at (5,0,5);
    \coordinate (ot) at (0,3,5);
    \coordinate (ob) at (0,-3,5);
    \coordinate (lb) at (-5,-3,5);
    \coordinate (rb) at (5,-3,5);
    \coordinate (lt) at (-5,3,5);
    \coordinate (rt) at (5,3,5);
    \draw [plane] (lb) -- (lt) -- (rt) -- (rb) -- cycle;
    \draw [dash] (ol) -- (or);
    \draw [dash] (ob) -- (ot);
    \node [xshift=3mm, yshift=5mm] at (lb) {$\mathit{\Omega}$};
    % Draw the rest of axis
    \draw [axis] (oo) node [below] {$\vec{c}$} -- (z) node [below] {$z$};
    
    % focal length
    \coordinate (of) at (-5,0,0);
   	\draw [dash] (O) -- (of);
    \draw [<->] (of) -- (ol) node [below] at (-5,0,2.5) {$f$};
    
    % delec
    \coordinate (p) at (5,3,10);
    \draw [fill=black] (p) circle (1.5mm) node [below] {$\vec{p}$};
    \draw [dash, name path=line 1] (O) -- (p);
    \draw [dash] (5,0,0) node [above] {$X$} -- (5,0,10);
    \draw [dash] (0,0,10) node [below] {$Z$} -- (5,0,10);
    \draw [dash] (5,0,10) -- (p);
    
    
    % slika delca
    \coordinate (q) at (2.5,1.5,5);
    \draw [fill=black] (q) circle (1mm) node [below] {$\vec{q}$};
    \draw [dash] (2.5,0,5) node [below] {$x$} -- (q);
    \draw [dash] (0,1.5,5) node [left] {$y$} -- (q);
   
   
\end{tikzpicture}
\caption{Perspektivni model kamere.}
\label{fig:perspektivni-model}
\end{figure}


\textbf{Koordinatni sistem kamere} (KSK) je postavljen tako, da \textbf{optična os} sovpada s $Z$ osjo. Zaradi poenostavitve KSK sovpada z globalnim koordinatnim sistemom.  Središče KSK $\vec{O}$ se imenuje projekcijsko središče, skozi katerega se točka prizora $\vec{p} = \left[ X~Y~Z \right]^\top$ projecira na slikovno ravnino \cite{trucco1998introductory}.

\textbf{Slikovna ravnina} (ang. image plane) je ravnina $\mathit{\Omega} \subset \mathbb{R}^2$, ki leži na razdalji $f$ od projekcijskega središča $O$. Razdalja $f$ se imenuje goriščna razdalja (ang. focal length) \cite{trucco1998introductory}. Točka $\vec{c} = \left[c_x~c_y \right]^\top$ se nahaja na poziciji, kjer optična os prebada slikovno ravnino $\mathit{\Omega}$. Imenuje se optično središče slikovne ravnine (ang. principal point). Točka $\vec{q} = [x~y]^\top$ se nahaja na poziciji, kjer daljica med projekcijskim središčem $\vec{O}$ in točko prizora $\vec{p}$ prebada slikovno ravnino $\mathit{\Omega}$ \cite{trucco1998introductory}. Točka je slika prizora $\vec{q} = [x~y]^\top$, kjer sta $x$ in $y$ slikovni koordinati. Kadar je $\vec{c} = \left[0~0\right]^\top$, lahko sliko $\vec{q}$ na slikovni ravnini $\mathit{\Omega}$ predstavimo z enačbo \eqref{eq:slika-delca}.

\begin{equation}
	\vec{q} = f \frac{\vec{p}}{Z}.
    \label{eq:slika-delca}
\end{equation}

\subsection{Diskretna slikovna ravnina}
Slikovne koordinate so v resnici diskretne, saj sliko sestavlja polje slikovnih elementov $\vec{\lambda}$ s širino $\lambda_u$ in dolžino $\lambda_v$ \cite{trucco1998introductory}. V splošnem je enačba slikovnih koordinat $x$ in $y$ v metričnih enotah \eqref{eq:slikovne-koordinate}, kjer sta $u$ in $v$ slikovni koordinati v pikslih. Točka $\vec{c} = \left[c_u~c_v \right]^\top$ je optično središče v pikslih.

\begin{subequations}
\begin{align}
	x &= \lambda_u (u - c_u) \\
    y &= \lambda_v (v - c_v)
\end{align}
\label{eq:slikovne-koordinate}
\end{subequations}

Če uporabimo homogene koordinate lahko ob upoštevanju enačb \eqref{eq:slika-delca} in \eqref{eq:slikovne-koordinate} izračunamo diskretne slikovne koordinate slike $\vec{q}' = \left[ u~ v \right]^\top$ po enačbi \eqref{eq:diskretne-slikovne-koordinate} \cite{trucco1998introductory}.


\begin{equation}
	w \begin{bmatrix}
	u \\ v \\ 1
	\end{bmatrix} = \vec{M}_{int}
    \begin{bmatrix}
    X \\ Y \\ Z \\ 1
    \end{bmatrix}
    \label{eq:diskretne-slikovne-koordinate}
\end{equation}

Matrika $\vec{M}_{int}$ v enačbi \eqref{eq:diskretne-slikovne-koordinate} je intrinsična matrika \cite{trucco1998introductory}. Predstavljena je z enačbo \eqref{eq:intrinsic} in vsebuje intrinzične parametre kamere, kjer sta $f_u$ in $f_v$ določena z enačbo \eqref{eq:focal}.

\begin{equation}
\vec{M}_{int} = \begin{bmatrix}
	f_u & 0 & c_u & 0 \\
    0 & f_v & c_v & 0 \\
    0 & 0 & 1 & 0
\end{bmatrix}
\label{eq:intrinsic}
\end{equation}

\begin{equation}
\begin{bmatrix}
	f_u & f_v
\end{bmatrix}^\top = \begin{bmatrix}
	\frac{f}{\lambda_u} & \frac{f}{\lambda_v}
\end{bmatrix}^\top
\label{eq:focal}
\end{equation}


\subsection{Premikanje kamere}
Koordinatni sistem kamere lahko transliramo in rotiramo tako, da ne sovpada več z globalnim koordinatnim sistemom, kot je prikazano na sliki \ref{fig:premikanje-kamere}. 

\begin{figure}[htb]
\centering
\begin{tikzpicture}%[tdplot_main_coords, scale=0.5]
[x={(0.8cm,0.4cm)}, y={(0cm,1cm)}, z={(0.8cm,-0.4cm)}, scale=0.5]

	% Coordinate system
    \coordinate (O) at (0,0,0);
    \coordinate (oo) at (0,0,5);
    \coordinate (y) at (0,5,0);
    \coordinate (z) at (0,0,15);
    \coordinate (x) at (10,0,0);
    \draw [axis] (O) -- (y) node [right] {$y$};
    \draw [base-axis] (O) -- (oo);
    \draw [axis] (O) -- (x) node [below] {$x$};
    \node (izhodisce) [below] at (O) {$\vec{O}$};
    
   
    
    
    % Image plane
    \coordinate (ol) at (-5,0,5);
    \coordinate (or) at (5,0,5);
    \coordinate (ot) at (0,3,5);
    \coordinate (ob) at (0,-3,5);
    \coordinate (lb) at (-5,-3,5);
    \coordinate (rb) at (5,-3,5);
    \coordinate (lt) at (-5,3,5);
    \coordinate (rt) at (5,3,5);
    \draw [plane] (lb) -- (lt) -- (rt) -- (rb) -- cycle;
    \draw [dash] (ol) -- (or);
    \draw [dash] (ob) -- (ot);
    \node [xshift=3mm, yshift=5mm] at (lb) {$\mathit{\Omega}$};
    % Draw the rest of axis
    \draw [axis] (oo) node [below] {$\vec{c}$} -- (z) node [below] {$z$};
    
    
    % Coordinate system2
    \begin{scope}{}
    \coordinate (O2) at (1,1,8);
    \coordinate (y2) at (6.7,1.4,6.5);
    \coordinate (z2) at (11.8,-5.4,7.5);
    \coordinate (x2) at (12.4,0.6,-3.1);
    \draw [axis, draw=orange!50!black] (O2) node [below] {$\vec{O}'$} -- (y2) node [above] {$x'$} node (x') [near start] {};
    \draw [axis, draw=orange!50!black] (O2) -- (x2) node [right] {$y'$};
    \draw [axis, draw=orange!50!black] (O2) -- (z2) node [below] {$z'$} node (z') [midway] {};
    \end{scope}
    
    
    
    
    % focal length
    \coordinate (of) at (-5,0,0);
   	\draw [dash] (O) -- (of);
    \draw [<->] (of) -- (ol) node [below] at (-5,0,2.5) {$f$};
    
    % delec
    \coordinate (p) at (5,3,10);
    \draw [fill=black] (p) circle (1.5mm) node [below] {$\vec{p}$};
    \draw [dash, name path=line 1] (O) -- (p);
    \draw [dash] (x') node [above] {$X'$} -- (5,0,10);
    \draw [dash] (z') node [below] {$Z'$} -- (5,0,10);
    \draw [dash] (5,0,10) -- (p);
    
    
    % slika delca
    \coordinate (q) at (2.5,1.5,5);
    \draw [fill=black] (q) circle (1mm) node [below] {$\vec{q}$};
    \draw [dash] (2.5,0,5) node [below] {$x$} -- (q);
    \draw [dash] (0,1.5,5) node [left] {$y$} -- (q);
   
   
\end{tikzpicture}
\caption{Koordinatni sistem kamere ne sovpada z globalnim koordinatnim sistemom}
\label{fig:premikanje-kamere}
\end{figure}

Translacijo koordinatnega sistema kamere lahko opišemo z vektorjem $\vec{t} = \left[t_x~t_y~t_z\right]^\top$ \cite{trucco1998introductory}. 

Rotacijo koordinatnega sistema kamere lahko opišemo z Eulerjevimi koti $\phi$ $\theta$ in $\psi$ \cite{bajd2011osnove}. S kotom $\phi$ rotiramo okoli $z$ osi (ang. Roll). Rotacija je predstavljena z enačbo \eqref{eq:roll}. Kot $\theta$ predstavlja rotacijo okoli $x$ osi (ang. Pitch). Rotacija je opisana z enačbo \eqref{eq:pitch}. Kot $\psi$ je rotacija okoli $y$ osi (ang. Yaw) in je opisana z enačbo \eqref{eq:yaw}.

\begin{equation}
\vec{R}_\phi = \begin{bmatrix}
\cos(\phi) & - \sin(\phi) & 0 \\
\sin(\phi) & \cos(\phi) & 0 \\
0 & 0 & 1
\end{bmatrix}
\label{eq:roll}
\end{equation}

\begin{equation}
\vec{R}_\theta = \begin{bmatrix}
1 & 0 & 0 \\
0 & \cos(\theta) & - \sin(\theta) \\
0 & \sin(\theta) & \cos(\theta)
\end{bmatrix}
\label{eq:pitch}
\end{equation}

\begin{equation}
\vec{R}_\psi = \begin{bmatrix}
\cos(\psi) & 0 & \sin(\psi) \\
0 & 1 & 0 \\
- \sin(\psi) & 0 & \cos(\psi) 
\end{bmatrix}
\label{eq:yaw}
\end{equation}


Kadar opravimo vse rotacije osi glede na fiksni globalni koordinatni sistem po vrstnem redu $\vec{R}_\psi$, $\vec{R}_\theta$ in $\vec{R}_\phi$, lahko rotacijsko matriko $\vec{R}$ opišemo z enačbo \eqref{eq:rotation} \cite{bajd2011osnove}.


\begin{equation}
\vec{R} = \vec{R}_\phi \vec{R}_\theta \vec{R}_\psi
\label{eq:rotation}
\end{equation}


Translacijo $\vec{t}$ in rotacijo $\vec{R}$ lahko združimo v matriko premika koordinatnega sistema kamere glede na globalni koordinatni sistem, ki jo imenujemo \textbf{ekstrinsična matrika} \cite{trucco1998introductory}. Opisana je z enačbo \eqref{eq:extrinsic}.

\begin{equation}
\vec{M}_{ext} = \begin{bmatrix}
	\vec{R} & \vdots & \vec{t}
\end{bmatrix}
\label{eq:extrinsic}
\end{equation}


Z upoštevanjem premika kamere lahko enačbo \eqref{eq:diskretne-slikovne-koordinate} zapišemo v obliko \eqref{eq:diskretne-slikovne-koordinate-premik}. Matrika $\vec{M}$ je \textbf{projekcijska matrika} in je opisana v enačbi \eqref{eq:projection-matrix}.

\begin{equation}
w \begin{bmatrix}
	u \\ v \\ 1
	\end{bmatrix} = \vec{M}
    \begin{bmatrix}
    X \\ Y \\ Z \\ 1
    \end{bmatrix}
\label{eq:diskretne-slikovne-koordinate-premik}
\end{equation}


\begin{equation}
\vec{M} = \vec{M}_{int} \vec{M}_{ext}
\label{eq:projection-matrix}
\end{equation}
