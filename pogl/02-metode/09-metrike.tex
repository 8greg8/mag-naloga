\section{Evaluacijske metrike}
V predstavljenih enačbah je $N$ število vzorcev, $\hat{y}_i~ \forall i=1,\ldots,N$ so izmerjene vrednosti in $y_i ~\forall i=1,\ldots,N$ so prave vrednosti. Spremenljivke s prečno črto predstavljajo srednjo vrednost, ki jo izračunamo po enačbi~\eqref{eq:mean}~\cite{witten2005data}.

\begin{equation}
\overline{y} = \frac{1}{N} \sum_{i=1}^N y_i
\label{eq:mean}
\end{equation}


\subsection{Koren srednje kvadratične napake}
Koren srednje kvadratične napake (angl. Root Mean Square Error) (\rmse) opišemo z enačbo~\eqref{eq:rmse}~\cite{witten2005data}.
\begin{equation}
e_{RMS} = \sqrt{\frac{1}{N} \sum_{i=1}^{N}\left( \hat{y}_i - y_i \right)^2}
\label{eq:rmse}
\end{equation}

Rezultati se gibljejo na intervalu $e_{RMS} \in [0, \infty)$. Manjša vrednost pomeni boljši rezultat~\cite{witten2005data}.

\subsection{Koren relativne kvadratične napake}
Koren relativne kvadratične napake (angl. Relative Root Squared Error) (\rrse) opišemo z enačbo~\eqref{eq:rmse}~\cite{witten2005data}. $\overline{y}$ je srednja vrednost pravih vrednosti, ki jo izračunamo po enačbi~\eqref{eq:mean}.

\begin{equation}
e_{RRS} = \sqrt{\frac{\sum_{i=1}^{N} \left ( \hat{y}_i - y_i \right )^2}{\sum_{i=0}^{N} \left( \overline{y} - y_i \right)^2}}
\label{eq:rrse}
\end{equation}

Rezultati se gibljejo na intervalu $e_{RRS} \in [0, \infty)$. Manjša vrednost pomeni boljši rezultat~\cite{witten2005data}.

Enačbo si lahko razlagamo kot \rmse matematičnega modela, ki je normaliziran na \rmse najenostavnejšega matematičnega modela ali prediktorja~\cite{witten2005data}. Najenostavnejši prediktor je tisti, ki nam na izhodu daje srednjo vrednost $\overline{y}$. 

\subsection{Relativna absolutna napaka}
Relativno absolutno napako (angl. Relative Absolute Error) (\rae) opišemo z enačbo~\eqref{eq:rmse}~\cite{trucco1998introductory}.

\begin{equation}
e_{RA} = \frac{\sum_{i=1}^{N} \left | \hat{y}_i - y_i \right |}{\sum_{i=1}^{N} \left| \overline{y} - y_i \right|}
\label{eq:rae}
\end{equation}

Rezultati se gibljejo na intervalu $e_{RAE} \in [0, \infty)$. Manjša vrednost pomeni boljši rezultat~\cite{witten2005data}.

\subsection{Korelacijski koeficient}
Korelacijski koeficient \corr je predstavljen z enačbo~\eqref{eq:r}~\cite{witten2005data}.

\begin{subequations}
\begin{align}
\corr &= \frac{S_{\hat{y}y}}{\sqrt{S_{\hat{y}} S_y}} \\
S_{\hat{y}y} &= \frac{\sum_{i=1}^N \left(\hat{y}_i - \overline{\hat{y}}\right)\left(y_i - \overline{y}\right) }{N-1} \\
S_{\hat{y}} &= \frac{\sum_{i=1}^N \left(\hat{y}_i - \overline{\hat{y}}\right)}{N-1} \\
S_y &= \frac{\sum_{i=1}^N \left(y_i - \overline{y}\right) }{N-1}
\end{align}
\label{eq:r}
\end{subequations}

Rezultati se gibljejo na intervalu $e_{RAE} \in [-1, 1]$~\cite{witten2005data}. Vrednost $-1$ pomeni, da so rezultati popolnoma korelirani s pravimi vrednostmi v nasprotni smeri. Vrednost $0$ pomeni, da ne obstaja nobena korelacija. Vrednost $1$ pomeni, da obstaja popolna korelacija.  


\subsection{Razmerje signal in šum}
Razmerje signal - šum (angl. Signal to Noise Ratio) (SNR) je določeno z enačbo~\eqref{eq:snr}, kjer je $P_s$ moč signala $y$ in $P_n$ moč šuma $n$~\cite{gonzalez2006digital}. Signal z nizko vsebnostjo šuma ima visok SNR, medtem ko nizek SNR pomeni veliko šuma v signalu.

\begin{subequations}
\begin{align}
SNR_{dB} &= 10 \log_{10}\left(\frac{P_s}{P_n}\right) \\
P_s &= \frac{1}{N} \sum_{i=1}^{N} y_i^2 \\
P_n &= \frac{1}{N} \sum_{i=1}^{N} n_i^2
\end{align}
\label{eq:snr}
\end{subequations}



\subsection{Mera prekrivanja področja}
Mera prekrivanja področja (angl. Region Overlap), ki je predstavljena z enačbo~\ref{eq:region-overlap}, se večinoma uporablja za sledilnike~\cite{vcehovin2016visual}. $\Lambda$ predstavlja opis tarče, $R_t$ je področje tarče ob času $t$, $N$ je število slik v zaporedju, $G$ predstavlja referenco in $T$ tarčo.

\begin{subequations}
\begin{align}
	\Lambda &= \left\{R_t\right\}^N_{t=1}, \nonumber \\
	\Phi(\Lambda^G, \Lambda^T) &= \left\{\phi_t\right\}^N_{t=1}, \nonumber \\
    \phi_t &= \frac{R_t^G \cap R_t^T }{R_t^G \cup R_t^T} 
\end{align}
\label{eq:region-overlap}
\end{subequations}


S prekrivanjem področja dobimo rezultate na intervalu $\left[0,1\right]$. Višja vrednost pomeni boljši rezultat.



