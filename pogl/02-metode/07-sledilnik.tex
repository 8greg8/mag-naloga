\section{Sledilnik}
% tracker
% Zakaj smo uporabili sledilnik
% Na katere sledilnike smo ciljali 
% Primerjava sledilnikov
% Naši eksperimenti kateri je boljši
%
Gibanje objektov v prostoru zaznamo kot temporalno spreminjanje slike. Ta lastnost nam omogoča izluščenje koristnih informacij, kot je identifikacija objektov na podlagi karakteristike gibanja, določevanje njihove pozicije in ugotavljanje kaj se v sceni dogaja \cite{forsyth2002computer}. Forsyth et. al \cite{forsyth2002computer} sledenje opiše kot postopek, s katerim sklepamo o gibanju objektov glede na zaporedje slik. Problem sledenja v računalniškem vidu še ni v celoti rešljiv, saj obstaja veliko faktorjev, ki otežujejo delo sledilnikov \cite{danelljan2014adaptive}. Med njimi sodijo okluzija objekta, ki mu sledimo, razne geometrijske deformacije in zamegljenost zaradi hitrega gibanja, temporalno spreminjanje osvetljenosti, šum iz ozadja ter variacije v skali. 

Sledlinike delimo v glavnem na dva pristopa: generativne in diskriminativne metode \cite{danelljan2014adaptive}. Pri \textbf{generativnih metodah} iščemo področja, ki so najbolj podobna modelu tarče. \textbf{Diskriminativne metode} predstavljajo binarni problem razvrščanja, saj pri njih želimo določiti mejo med tarčo in ozadjem. Slednjo metodo uporabljamo pri sledenju z detekcijo, ki daje najboljše rezultate \cite{danelljan2014adaptive}. Glavna ideja takega sledenja je sprotno učenje razvrščevalnika s trenutnim vzorcem tarče. Sledi korak detekcije tarče na naslednji sliki zaporedja z razvrščevalnikom in vizualno predstavitvijo tarče, ki smo se jo naučili skozi čas.






\subsection{Zmanjševanje merilne napake}

V našem delu opazujemo gibajoča telesa, zato je smiselno vpeljati sledilnik v merjenje energijske porabe. Z uporabo sledilnika dobimo realnejšo sliko meritve, saj se znebimo šuma iz ozadja (premikanja slikovnih elementov, ki niso del tarče). Izognemo se napaki merjenja zaradi šuma CCD senzorja in premikanja drugih objektov. To so lahko razni predmeti (žoge, loparji, itd.) ali druge osebe. Brezkontaktnemu merilnemu instrumentu tako dodamo širšo uporabno vrednost, saj ga v nasprotnem primeru ne bi mogli uporabljati v ekipnih športih in športih z žogo. 

Zaradi uporabe optičnega in prostorskega toka pri določevanju modela gibanja nam merilno napako povzroča tudi premikanje kamere. S premikanjem kamere povzročimo relativno premikanje objetkov glede na koordinatni sistem kamere, četudi so ti glede na referenčni koordinatni sistem v prostoru pri miru. Tega problema se lahko znebimo z uporabo sledilnika. 

Predpostavimo, da imamo idealni sledilnik in enako definicijo in omejitve kamere, masnega delca, slike delca in osvetlitve kot v poglavju \ref{sec:opticni-tok}. Idealni sledilnik v vsaki sliki zaporedja najde sliko delca $\vec{q}$ in posamezno sliko v zaporedju obreže tako, da je težišče tarče vedno v centru obrezane slike. Ne glede na gibanje kamere, bo pozicija slike delca $\vec{q}$ vedno v centru slikovne ravnine. Gibanje kamere zato ne bo vplivalo na optični tok $\mathcal{O}$.



\subsection{Sledilnik za optični tok} 
Sledilnik temelji na delu \cite{danelljan2014adaptive}, kjer izboljšajo originalni KCF sledilnik iz dela \cite{henriques2012exploiting} z uporabo barvnih značilk. KCF sledilnik sodi med metode sledenja z detekcijo. Njegovo delovanje je prikazano na sliki \ref{fig:diagram-kcf}.

\paragraph{Korak učenja.}
Na področju tarče $x$ velikosti $M \times N$, vsakemu slikovnemu elementu pripada vrednost Gaussove funkcije $y$. Ob času $p$ poznamo $\mathcal{X}^p = \{x^j: j=1,\ldots,p\}$ področij tarče. Razvrščevalnik učimo z minimizacijo funkcije \eqref{eq:classifier-function}, ki predstavlja uteženo srednjo kvadratično napako čez področja $\mathcal{X}^p$, pri čemer je vsako področje uteženo s konstanto $\beta_j \geq 0$. $\phi$ predstavlja neliearno preslikavo v več rasežni prostor, za katero lahko uporabimo implicitno preslikavo ali funkcijo jedra $K$. V KCF sledilniku se uporablja Gaussovo RBF jedro \eqref{eq:kcf-gauss}. Konstanta $\lambda \geq 0$ je regularizacijski parameter.

\begin{equation}
\begin{aligned}
\epsilon &= \sum_{j=1}^p \beta_j \left( 
	\sum_{m,n} \left| \langle \phi\left(x_{m,n}^j \right), w^j \rangle - y^j(m,n) \right|^2
    + \lambda \langle w^j, w^j \rangle
\right), \\
w^j &= \sum_{k,l} a(k,l) \phi\left(x_{k,l}^j  \right)
\end{aligned}
\label{eq:classifier-function}
\end{equation}

V enačbi \eqref{eq:kcf-gauss} je $\mathcal{F}$ operator diskretne Fourierove transformacije (DFT). Velike začetnice spremenljivk predstavljajo njihove DFT. Tako je $X = \mathcal{F}\{ {x} \}$ DFT področja $x$. $\sigma$ je hiperparameter jedra.

\begin{equation}
K_{RBF}({x}, {x}') = \mathrm{e}^{-\frac{1}{\sigma^2}\left(
	||{x}||^2 + ||{x}'||^2 - 2 \mathcal{F}^{-1}\left\{ {X} {X}' \right\}
\right)}
\label{eq:kcf-gauss}
\end{equation}

Funkcijo \eqref{eq:classifier-function} minimiziramo s koeficientom \eqref{eq:classifier-a}.  $Y^p = \mathcal{F}\{y^p\}$ je DFT gaussove funkcije in $U_x^p = \mathcal{F}\{ K(x_{m,n}^p, x^p) \}$ je DFT jedrne funkcije $K$. $\gamma$ je parameter učenja.

\begin{equation}
A^p = \mathcal{F}\{a^p\} =  \frac{(1- \gamma) A_N^{p-1} + \gamma Y^p U_x^p}
{(1- \gamma)A_D^{p-1} + \gamma U_x^p\left( U_x^p + \gamma \right)}
\label{eq:classifier-a}
\end{equation}

Naučeno vizualno podobo tarče $\hat{x}^p$ ob času $p$ posodobimo z enačbo \eqref{eq:training-target}.

\begin{equation}
\hat{x}^p = (1 - \gamma) \hat{x}^{p-1} + \gamma x^p
\label{eq:training-target}
\end{equation}


\paragraph{Korak detekcije.}
Pri detekciji najprej izrežemo področje $z$ velikosti $M \times N$ na novi sliki. Nato izračunamo rezultate detekcije po enačbi \eqref{eq:detection-score}, kjer je $U_z = \mathcal{F}\left\{ K\left( z_{m,n}, \hat{x}^{p}  \right) \right\}$ DFT izhoda jedrne funkcije področja $z$.

\begin{equation}
\hat{y}^{p + 1} = \mathcal{F}^{-1}\left\{ A U_z \right\}
\label{eq:detection-score}
\end{equation}

Pozicijo tarče nato dobimo s tisto translacijo, ki maksimizira rezultat detekcije $\hat{y}^{p+1}$.




\begin{figure}[htb]
\centering
\begin{tikzpicture}
% LAYERS
\pgfdeclarelayer{bg}
\pgfsetlayers{bg,main}
\tikzset{
    between/.style args={#1 and #2}{
         at = ($(#1)!0.5!(#2)$)
    }
}

% NODES
\node (slika) [input] at (0,0) {Področje\\tarče $x^{p}$};
\node (ucenje) [block, right= of slika] {Učenje razvrščevalnika};
\node (tarca) [block, below= of ucenje] {Posodobitev podobe\\tarče $\hat{x}^p$};


\node (podrocje) [block, right=1cm of tarca] {Izrezano področje\\slike $z$};
\node (razvrscanje) [block, above= of podrocje] {Razvrščanje področja};
\node (prcenter) [between=razvrscanje.south and podrocje.north] {};
\node (iskanje) [block, right=2cm of prcenter] {Iskanje tarče};


\node (center) [between= podrocje.west and iskanje.east] {};
\node (kd) [title, above=2cm of center] {Korak detekcije};

\coordinate (c) at ( ucenje.center |- kd);
\node (ku) [title] at (c) {Korak učenja};


\node (rezultat) [output, right= of iskanje] {Novo področje\\ tarče $x^{p+1}$};

% arrows
\draw [arrow] (slika) -- (ucenje);
\draw [arrow] (ucenje) -- (tarca);
\draw [arrow] (tarca) -- (podrocje);
\draw [arrow] (podrocje) -- (razvrscanje);
\draw [arrow] (razvrscanje.east) -- (iskanje.north west);
\draw [arrow] (iskanje.south west) -- (podrocje.north east);
\draw [arrow] (iskanje) -- (rezultat);

\begin{pgfonlayer}{bg}
	\node (ltop) [above = 1mm of ku] {};
	\node (lleft) [left = 1mm of tarca] {};
	\node (lbottom) [below= 1mm of tarca] {};
    \node (lright) [right= 1mm of tarca] {};
  	\node (l0) [] at ( lbottom -| lleft) {};
    \node (l1) [] at (ltop -| lright) {};
    \path[background] (l0) rectangle (l1);
    
    \node (rtop) [above = 1mm of kd] {};
    \node (rright) [right = 1mm of iskanje] {};
    \node (rbottom) [below = 1mm of podrocje] {};
    \node (rleft) [left = 1mm of podrocje] {};
    \node (r0) at (rbottom -| rleft) {};
    \node (r1) at (rtop -| rright) {};
    \path [background] (r0) rectangle (r1);
\end{pgfonlayer}
\end{tikzpicture}
\caption[Diagram KCF sledilnika]{Diagram KCF sledilnika.}
\label{fig:diagram-kcf}
\end{figure}







\subsection{Sledilnik za prostorski tok}
Jedro sledilnika temelji na KCF sledilniku za optični tok iz dela \cite{henriques2015high}. Pri tem uporabljajo jedrno funkcijo \eqref{eq:kcf-gauss}. Model tarče je predstavljen z vektorjem značilk, ki je sestavljen iz histograma orientiranih gradientov (HOG) barvne slike in HOG histograma globinske slike. 

V DS-KCF sledilniku Najprej segmentirajo globinsko sliko na področja podobne globine s pomočjo rojenja \cite{hannuna2016ds}. S tem pridobijo relevantne značilke globinske distribucije. 

S pomočjo globinske distribucije izračunajo spremembe v skali glede na začetno srednjo vrednost globine tarče in jih uporabijo za posodobitev modela tarče. Posodobitev poteka, ko dobijo novo estimacijo pozicije tarče. Pri tem uporabljajo interpolacijo ali decimacijo v frekvenčnem prostoru \cite{hannuna2016ds}.

V istem času ko sledilnik računa spremembe v skali se globinska distribucija uporabi tudi za detekcijo okluzij. Kadar sledilnik detektira, da je prišlo do okluzije se model tarče ne posodobi \cite{hannuna2016ds}. Pri določevanju okluzije sledilnik uporablja Kalmanov filter, s katerim sledi centru tarče in objekta, ki povzroča okluzijo. V \cite{hannuna2016ds} uporabljajo linearni model konstantne hitrosti.  

Na koncu sledijo popravki zaradi sprememb oblike objekta. Popravki temeljijo na razmerju stranic začetnega pravokotnika modela tarče \cite{hannuna2016ds}. Sledilnik model tarče popravi vedno tako, da razmerje stranic ostaja konstantno.




\begin{figure}[htb]
\centering
\begin{tikzpicture}
\node (depth) [input] at (0,0) {Globinska slika};
\node (rgb) [input, below= of depth] {RGB slika};
\node (segment) [block, minimum width=1cm, right= of depth] {Segmentacija globinske slike};
\node (feature) [block, below= of segment] {Ekstrakcija značilk};
\node (kcf) [block, minimum width=4cm, minimum height=2cm, right=2cm of feature] {KCF};
\node (scale) [block, above left=5mm of kcf.north] {Analiza skale};
\node (shape) [block, above right=5mm of kcf.north] {Analiza oblike};
\node (occlusion) [block, below left=5mm of kcf.south] {Upravljanje z okluzijo};
\node (kalman) [block, below right=5mm of kcf.south] {Kalmanov filter};

% arrows
\draw [vec] (depth) -- (segment);
\draw [vec] (rgb) -- (feature);
\draw [vec] (depth.south east) -- (feature.north west);

\draw [vec] (segment) -- (scale);
\draw [vec] (segment.north east) to[out=45, in=90] (shape.north);
\draw [vec] (segment.south east) -- (occlusion);

\draw [vec] (feature) -- (kcf);

\draw [vec, <->] (scale) -- (kcf);
\draw [vec, <->] (shape) -- (kcf);
\draw [vec, <->] (occlusion) -- (kcf);
\draw [vec, <->] (kalman) -- (kcf);
\draw [vec, <->] (occlusion) -- (kalman);
\end{tikzpicture}
\caption{Diagram DS-KCF sledilnika po \cite{hannuna2016ds}.}
\label{fig:diagram-dskcf}
\end{figure}
