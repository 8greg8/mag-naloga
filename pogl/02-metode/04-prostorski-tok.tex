\section{Prostorski tok}
% Teorija prostorskega toka
% Katero metodo smo mi uporabili in zakaj
Optični tok $\vec{O}$ predstavlja aproksimacijo polja gibanja $\vec{G}$, ta pa je projekcija polja hitrosti $\vec{H}$ na slikovno ravnino $\mathit{\Omega}$ \cite{trucco1998introductory}. Če pogledamo z druge perspektive, ni optični tok $\vec{O}$ nič drugega kot projekcija aproksimacije polja hitrosti $\vec{H}$, ki jo po analogiji lahko imenujemo \textbf{prostorski tok} (angl. Scene Flow) \cite{vedula1999three}. 

Za namen razlage upoštevamo enake omejitve kamere, masnega delca in osvetlitve, kot v poglavju \ref{sec:opticni-tok}. Predpostavimo, da imamo v prostoru površino $f(x,y,z) = 0$ na kateri imamo gibajoč točkovni delec $\vec{p} = \vec{p}(t)$. Na slikovni ravnini $\mathit{\Omega}$ imamo njegovo sliko $\vec{q}$ \cite{vedula1999three}. Vizualni prikaz prizora lahko vidimo na sliki \ref{fig:scene-flow}. 




\begin{figure}[htb]
\centering
\begin{tikzpicture}%[tdplot_main_coords, scale=0.5]
[x={(0.8cm,0.4cm)}, y={(0cm,1cm)}, z={(0.8cm,-0.4cm)}, scale=0.5]

	% Coordinate system
    \coordinate (O) at (0,0,0);
    \coordinate (oo) at (0,0,5);
    \coordinate (y) at (0,5,0);
    \coordinate (z) at (0,0,15);
    \coordinate (x) at (10,0,0);
    \draw [axis] (O) -- (y) node [above] {$y$};
    \draw [base-axis] (O) -- (oo);
    \draw [axis] (O) -- (x) node [below] {$x$};
    \node (izhodisce) [below] at (O) {$o$};
    
    % Image plane
    \coordinate (ol) at (-5,0,5);
    \coordinate (or) at (5,0,5);
    \coordinate (ot) at (0,3,5);
    \coordinate (ob) at (0,-3,5);
    \coordinate (lb) at (-5,-3,5);
    \coordinate (rb) at (5,-3,5);
    \coordinate (lt) at (-5,3,5);
    \coordinate (rt) at (5,3,5);
    \draw [plane] (lb) -- (lt) -- (rt) -- (rb) -- cycle;
    \draw [dash] (ol) -- (or);
    \draw [dash] (ob) -- (ot);
    \node [xshift=3mm, yshift=5mm] at (lb) {$\mathit{\Omega}$};
    % Draw the rest of axis
    \draw [axis] (oo) -- (z) node [below] {$z$};
    
    % focal length
    \coordinate (of) at (-5,0,0);
   	\draw [dash] (O) -- (of);
    \draw [<->] (of) -- (ol) node [below] at (-5,0,2.5) {$f$};
    
    % povrsina
    \begin{scope}
    \coordinate (p0) at (3,1,12);
    \coordinate (p1) at (3,5,12);
    \coordinate (p2) at (7,5,12);
    \coordinate (p3) at (7,1,12);
   
    \draw [plane, fill=purple!20, draw=purple!50!black!50,] 
    			   (p0) .. controls (3,4,10) .. (p2)
                        .. controls (7,3,10) .. (p3) 
                        .. controls (5,1,10) .. cycle;
    \node at (6,2,12) {$f$};
    \end{scope}
    
    
    
    % delec
    \coordinate (p) at (5,3,10);
    \draw [fill=black] (p) circle (1.5mm) node [below] {$\vec{p}$};
    \draw [dash, name path=line 1] (O) -- (p);
    %\draw [dash] (5,0,0) node [above] {$X$} -- (5,0,10);
    %\draw [dash] (0,0,10) node [below] {$Z$} -- (5,0,10);
    %\draw [dash] (5,0,10) -- (p);
    
    % hitrost
    \coordinate (v) at (6,4,10);
    \draw [velocity] (p) -- (v) node [above] {$\vec{\mu}$};
    \draw [dash] (O) -- (v);
    
    % slika delca
    \coordinate (q) at (2.5,1.5,5);
    \draw [fill=black] (q) circle (1mm) node [below] {$\vec{q}$};
    \draw [dash] (2.5,0,5) node [below] {$x$} -- (q);
    \draw [dash] (0,1.5,5) node [left] {$y$} -- (q);
    
    %hitrost
    \draw [velocity] (q) -- (3,2,5) node [above] {$\vec{w}$};
\end{tikzpicture}
\caption[Vizualni prikaz vektorja prostorskega toka $\vec{w}$]{Vizualni prikaz vektorja prostorskega toka $\vec{w}$. V prostoru imamo površino $f$ na kateri leži gibajoči točkovni delec $\vec{p}$ \cite{vedula1999three}. Na slikovni ravnini $\mathit{\Omega}$ imamo sliko delca $\vec{q}$ s prostorskim tokom $\vec{w}$.}
\label{fig:scene-flow}
\end{figure}



Ker je slika projekcija delca na slikovno ravnino $\mathit{\Omega}$, lahko zapišemo $\vec{q} = \vec{q}(\vec{p})$. Hitrost slike določimo po enačbi \eqref{eq:opticni-tok-sf}, ki predstavlja enačbo vektorja optičnega toka $\vec{w}$, kot projekcijo prostorskega toka \cite{vedula1999three}. 


\begin{equation}\label{eq:opticni-tok-sf}
	\vec{w} = \frac{d\vec{q}}{dt} = \frac{\partial \vec{q}}{\partial \vec{p}}\frac{d\vec{p}}{dt}
\end{equation}

Če predpostavimo, da imamo dovoj informacije o sistemu, da lahko določimo inverzno funkcijo $\vec{p} = \vec{p}(\vec{q},t)$, kjer je masni delec $\vec{p}$ projekcija slike $\vec{q}$, lahko določimo njegovo hitrost $\vec{\mu} \in \mathcal{\Mu} \subset \mathbb{R}^3$ z enačbo \eqref{eq:scene-flow}. Slednja je sestavljena iz dveh delov. Prvi člen je projekcija vektorja optičnega toka $\vec{w}$ na tangentno ravnino površine $f$, v točki, kjer se nahaja delec $\vec{p}$ \cite{vedula1999three}. Drugi člen je hitrost spreminjanja oddaljenosti delca od slikovne ravnine $\mathit{\Omega}$, ko slika delca stoji na miru. 

\begin{equation}\label{eq:scene-flow}
	\vec{\mu} = \frac{d\vec{p}}{dt} = \frac{\partial \vec{p}}{\partial \vec{q}} \frac{d\vec{q}}{dt} + \left.\frac{\partial \vec{p}}{\partial t}\right|_\vec{q}
\end{equation}

Kadar imamo v prostoru več, med seboj neodvisnih premikajočih masnih delcev, vektorsko polje hitrosti $\vec{\mu}$ imenujemo prostorski tok (angl. Scene Flow) $\vec{S}: \mathcal{W} \times \mathbb{R} \to \mathcal{\Mu}$, kjer velja $(\vec{w}, \dot{Z}) \mapsto \vec{\mu}$ \cite{yan2016scene}. $\dot{Z}$ predstavlja hitrost spreminjanja globine.

\subsection{Metode estimacije prostorskega toka}
Konvencionalne metode estimacije prostorskega toka so se razvile iz optičnega toka, in dodatne informacije o globini \cite{yan2016scene}. Slednjo lahko pridobimo s parom stereo kamer ali z uporabo sistemov večih kamer \cite{jaimez2015primal}. Vektor hitrosti prostorskega toka lahko v takih sistemih aproksimiramo z $\vec{\mu} = \left[w_x~w_y~\dot{d}\right]^\top$, kjer sta $(w_x, w_y)$ komponenti vektorja optičnega toka $\vec{w}$, $\dot{d}$ pa časovna sprememba disparitete \cite{yan2016scene}.

Z razvojem RGB-D kamer, kjer RGB predstavlja barvno sliko, D pa globinsko sliko, smo dobili cenovno dostopne in natančne sisteme, ki omogočajo implementacijo hitrih algoritmov prostorskega toka \cite{yan2016scene,jaimez2015primal}. Ti večinoma temeljijo na globalni variacijski metodi, kjer rešujemo minimizacijski problem 

\begin{equation}\label{eq:minimizacijski-problem}
	\min_{\mu}\{E_D(\vec{\mu}) + E_R(\vec{\mu})\}
\end{equation}

V podatkovnem delu funkcionala \eqref{eq:minimizacijski-problem} $E_D(\vec{\mu})$ \eqref{eq:podatkovni-del} upoštevamo konstantno osvetljenost \eqref{eq:konstantna-osvetljenost} in konsistentnost spreminjana globine \eqref{eq:konsistentnost-globine}, kjer je  $\Psi$ cenilka. Ponavadi je uporabljena $L_2$ norma $\Psi(x) = \| x \|^2$ \cite{yan2016scene}. $\alpha$ v \eqref{eq:podatkovni-del} predstavlja utež.

\begin{equation}\label{eq:konstantna-osvetljenost}
	E_{KO} = \sum_\Omega \Psi( I(\vec{q} + \vec{w}) - I(\vec{q}))
\end{equation}

\begin{equation}\label{eq:konsistentnost-globine}
	E_{KG} = \sum_\Omega \Psi\left( Z(\vec{q} + \Delta \vec{q}) - Z(\vec{q}) - \dot{Z}(\vec{q}))\right)
\end{equation}

\begin{equation}\label{eq:podatkovni-del}
	E_D = E_{KO} + \alpha E_{KG}
\end{equation}

V regularizacijskem delu funkcionala \eqref{eq:minimizacijski-problem} $E_R(\vec{\mu})$ pa uporabimo enačbo \eqref{eq:regularizacijski-del}

\begin{equation}\label{eq:regularizacijski-del}
	E_R = \sum_\Omega \Psi\left( \nabla w_x \right) + \Psi\left( \nabla w_y \right) + \Psi\left( \nabla \dot{Z} \right)
\end{equation}

V nadaljevanju predstavljamo globalno variacijsko metodo PD-Flow algoritem.


\subsubsection{PD-Flow algoritem.}\label{sec:pd-flow}
Algoritem spada pod globalne variacijske metode \cite{jaimez2015primal}. Za cenilko $\Psi$ v \eqref{eq:konstantna-osvetljenost} in \eqref{eq:konsistentnost-globine} avtorji uporabljajo $L_1$ normo. Za utež $\alpha$ v podatkovnem delu funkcionala \eqref{eq:podatkovni-del} se v tem algorimu uporablja funkcija 

\begin{equation}\label{eq:utez}
 \alpha(x,y) = \frac{\mu_0}{1 + k_\mu \left( \frac{\partial Z^2}{\partial x} + \frac{\partial Z^2}{\partial y} + \frac{\partial Z^2}{\partial t} \right)},
\end{equation}

kjer sta empirično določena parametra $\mu_0 = 75$ in $k_mu = 1000$. Za izračun podatkovnega dela uporabljajo hierarhično metodo grajenja slikovne piramide, pri tem pa uporabljajo linearizacijo podatkovnega dela \eqref{eq:podatkovni-del} \cite{jaimez2015primal}.

Jaimez et al. v delu \cite{jaimez2015primal} predstavi nov regularizacijski del \eqref{eq:regularizacijski-del-pdflow}, kjer upošteva še geometrijo prizora s faktorjem $\vec{r}$ \eqref{eq:faktor-prizora}. Z njim upošteva, da lahko sosednji slikovni elementi predstavljajo točke v prostoru, ki so si različno oddaljene.

\begin{equation}\label{eq:regularizacijski-del-pdflow}
E_R = \sum_\Omega \Psi\left( (\nabla w_x)^\top \vec{r} \right) + \Psi\left( (\nabla w_y)^\top \vec{r} \right) + \Psi\left( (\nabla \dot{Z})^\top \vec{r} \right)
\end{equation}

\begin{equation}\label{eq:faktor-prizora}
\vec{r} =
\begin{bmatrix}
\frac{1}{\sqrt{\frac{\partial X^2}{\partial x} + \frac{\partial Z^2}{\partial x}}} &
\frac{1}{\sqrt{\frac{\partial Y^2}{\partial y} + \frac{\partial Z^2}{\partial y}}}
\end{bmatrix}^\top
\end{equation}

Evaluacija PD-Flow algoritma je prikazana v tabeli \ref{tab:pdflow}. V tabeli so zapisani še rezultati RGB-D flow algoritma, ki za cenilko $\Psi$ ravno tako uporablja $L_1$ normo \cite{jaimez2015primal}. Opazimo lahko, da se PD-Flow po metrikah bolje odnese. Največji izboljšanje vidimo pri času izvajanja algoritma.

\begin{table}[htb]
	\centering
    \begin{tabular}{l S[table-format=1.3] S[table-format=2.3] S[table-format=3.3] S[table-format=3.3]}
    \toprule
    \textbf{Algoritem} & \multicolumn{1}{c}{\textbf{NRMS-V}} & \multicolumn{1}{c}{\textbf{AAE}} & \multicolumn{1}{c}{\textbf{Čas izvajanja [s]}} & \multicolumn{1}{c}{\textbf{MAX-V [m]}} \\
    \midrule
    \textbf{PD-Flow} & \boldentry{1.3}{0.068} & \boldentry{2.3}{6.653} & \boldentry{3.3}{7.150} & 0.111 \\
    RGB-D flow & 0.096 & 15.58 & 119.1 & 0.111 \\
    \bottomrule
    \end{tabular}
    \caption[Evaluacija PD-Flow algoritma]{Evaluacija PD-Flow algoritma in primerjava z algoritmom RGB-D flow, ki uporablja enako cenilko $\Psi$ \cite{jaimez2015primal}. Za metrike se uporabljata povprečna kotna napaka (AAE) in normaliziran koren srednje kvadratične napake magnitude hitrosti (NRMS-V), kjer se največja magnituda (MAX-V) uporablja za normalizacijo \cite{jaimez2015primal}. Opazimo lahko, da se PD-Flow po metrikah bolje odnese. Največji izboljšanje vidimo pri času izvajanja algoritma. Odebeljene vrednosti predstavljajo najboljšo vrednost.}
    \label{tab:pdflow}
\end{table}
