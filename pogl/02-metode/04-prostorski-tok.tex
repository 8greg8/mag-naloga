\section{Prostorski tok}
% Teorija prostorskega toka
% Katero metodo smo mi uporabili in zakaj
Optični tok $\vec{O}$ predstavlja aproksimacijo polja gibanja $\vec{G}$, ta pa je projekcija polja hitrosti $\vec{H}$ na slikovno ravnino $\varOmega$~\cite{trucco1998introductory}. Če pogledamo z druge perspektive, je optični tok $\vec{O}$ pravzaprav projekcija aproksimacije polja hitrosti $\vec{H}$, ki jo po analogiji lahko imenujemo \emph{prostorski tok} (angl. Scene Flow)~\cite{vedula1999three}.

Za namen razlage upoštevamo enake omejitve kamere, masnega delca in osvetlitve, kot v poglavju~\ref{sec:opticni-tok}. Predpostavimo, da imamo v prostoru površino $f(x,y,z) = 0$, na kateri imamo gibajoč točkovni delec $\vec{p} = \vec{p}(t)$. Na slikovni ravnini $\varOmega$ imamo njegovo sliko $\vec{q}$~\cite{vedula1999three}. Vizualni prikaz prizora lahko vidimo na sliki~\ref{fig:scene-flow}. 




\begin{figure}[htb]
\centering
\input{./Slike/scene-flow.tikz}
\caption[Vizualni prikaz vektorja prostorskega toka $\vec{w}$]{Vizualni prikaz vektorja prostorskega toka $\vec{w}$. V prostoru imamo površino $f$, na kateri leži gibajoči se točkovni delec $\vec{p}$~\cite{vedula1999three}. Na slikovni ravnini $\varOmega$ imamo sliko delca $\vec{q}$ s prostorskim tokom $\vec{w}$.}
\label{fig:scene-flow}
\end{figure}



Ker je slika projekcija delca na slikovno ravnino $\varOmega$, lahko zapišemo $\vec{q} = \vec{q}(\vec{p})$. Hitrost slike določimo po enačbi~\eqref{eq:opticni-tok-sf}, ki predstavlja enačbo vektorja optičnega toka $\vec{w}$ kot projekcijo prostorskega toka~\cite{vedula1999three}. 


\begin{equation}\label{eq:opticni-tok-sf}
	\vec{w} = \frac{d\vec{q}}{dt} = \frac{\partial \vec{q}}{\partial \vec{p}}\frac{d\vec{p}}{dt}
\end{equation}

Če predpostavimo, da imamo dovolj informacije o sistemu, da lahko določimo inverzno funkcijo $\vec{p} = \vec{p}(\vec{q},t)$, kjer je masni delec $\vec{p}$ projekcija slike $\vec{q}$, lahko določimo njegovo hitrost $\vec{\mu} \in \mathcal{\Mu} \subset \mathbb{R}^3$ z enačbo~\eqref{eq:scene-flow}. Slednja je sestavljena iz dveh delov. Prvi člen je projekcija vektorja optičnega toka $\vec{w}$ na tangentno ravnino površine $f$ v točki, kjer se nahaja delec $\vec{p}$~\cite{vedula1999three}. Drugi člen je hitrost spreminjanja oddaljenosti delca od slikovne ravnine $\varOmega$, ko slika delca stoji na miru. 

\begin{equation}\label{eq:scene-flow}
	\vec{\mu} = \frac{d\vec{p}}{dt} = \frac{\partial \vec{p}}{\partial \vec{q}} \frac{d\vec{q}}{dt} + \left.\frac{\partial \vec{p}}{\partial t}\right|_{\vec{q}}
\end{equation}

Kadar imamo v prostoru več med seboj neodvisnih premikajočih se masnih delcev, vektorsko polje hitrosti $\vec{\mu}$ imenujemo prostorski tok (angl. Scene Flow) $\vec{S}: \mathcal{W} \times \mathbb{R} \to \mathcal{\Mu}$, kjer velja

\begin{equation}
(\vec{w}, \dot{Z}) \mapsto \vec{\mu}
\label{eq:of-sf}
\end{equation}



$\dot{Z}$ predstavlja hitrost spreminjanja globine~\cite{yan2016scene}.

\subsection{Metode estimacije prostorskega toka}
Konvencionalne metode estimacije prostorskega toka so se razvile iz optičnega toka in dodatne informacije o globini~\cite{yan2016scene}. Slednjo lahko pridobimo s stereo parom kamer ali z uporabo sistemov več kamer~\cite{jaimez2015primal}. Vektor hitrosti prostorskega toka lahko v takih sistemih aproksimiramo z $\vec{\mu} = \left[w_x~w_y~\dot{d}\right]^\top$, kjer sta $(w_x, w_y)$ komponenti vektorja optičnega toka $\vec{w}$, $\dot{d}$ pa časovna sprememba disparitete~\cite{yan2016scene}.

Z razvojem kamer, ki temeljijo na principu merjenja časa preleta (angl. ``Time-of-flight'', ToF) smo dobili cenovno dostopne in natančne sisteme, ki omogočajo implementacijo hitrih algoritmov prostorskega toka~\cite{yan2016scene,jaimez2015primal}. Ti večinoma temeljijo na globalni variacijski metodi, kjer rešujemo minimizacijski problem 

\begin{equation}\label{eq:minimizacijski-problem}
	\min_{\mu}\{E_D(\vec{\mu}) + E_R(\vec{\mu})\}.
\end{equation}

Podatkovni del minimizacijskega problema je označen z $E_D(\vec{\mu})$, regularizacijski del pa z  $E_R(\vec{\mu})$. Podatkovni del izračunamo po enačbi~\eqref{eq:podatkovni-del}, kjer upoštevamo konstantno osvetljenost po~\eqref{eq:konstantna-osvetljenost} in konsistentnost spreminjanja globine po enačbi~\eqref{eq:konsistentnost-globine}. V slednji je  $\Psi$ cenilka. Običajno je uporabljena $L_2$ norma $\Psi(x) = \| x \|^2$~\cite{yan2016scene}. $\alpha$ v~\eqref{eq:podatkovni-del} predstavlja utež.

\begin{equation}\label{eq:konstantna-osvetljenost}
	E_{KO} = \sum_\Omega \Psi( I(\vec{q} + \vec{w}) - I(\vec{q}))
\end{equation}

\begin{equation}\label{eq:konsistentnost-globine}
	E_{KG} = \sum_\Omega \Psi\left( Z(\vec{q} + \Delta \vec{q}) - Z(\vec{q}) - \dot{Z}(\vec{q}))\right)
\end{equation}

\begin{equation}\label{eq:podatkovni-del}
	E_D = E_{KO} + \alpha E_{KG}
\end{equation}

Regularizacijski del $E_R(\vec{\mu})$ izračunamo po enačbi~\eqref{eq:regularizacijski-del}

\begin{equation}\label{eq:regularizacijski-del}
	E_R = \sum_\Omega \Psi\left( \nabla w_x \right) + \Psi\left( \nabla w_y \right) + \Psi\left( \nabla \dot{Z} \right)
\end{equation}

V nadaljevanju predstavljamo globalno variacijsko metodo PD-Flow.


\subsubsection{Algoritem PD-Flow}\label{sec:pd-flow}
Algoritem spada pod globalne variacijske metode~\cite{jaimez2015primal}. Za cenilko $\Psi$ v~\eqref{eq:konstantna-osvetljenost} in~\eqref{eq:konsistentnost-globine} avtorji uporabljajo normo $L_1$. Za utež $\alpha$ v enačbi~\eqref{eq:podatkovni-del} uporabljajo funkcijo 

\begin{equation}\label{eq:utez}
 \alpha(x,y) = \frac{\mu_0}{1 + k_\mu \left( \frac{\partial Z^2}{\partial x} + \frac{\partial Z^2}{\partial y} + \frac{\partial Z^2}{\partial t} \right)},
\end{equation}

kjer sta empirično določena parametra $\mu_0 = 75$ in $k_\mu = 1000$. Za izračun enačbe~\eqref{eq:podatkovni-del} uporabljajo hierarhično metodo grajenja slikovne piramide~\cite{jaimez2015primal}.

Jaimez et al. v delu~\cite{jaimez2015primal} predstavi nov regularizacijski del~\eqref{eq:regularizacijski-del-pdflow}, kjer upošteva še geometrijo prizora s faktorjem $\vec{r}$~\eqref{eq:faktor-prizora}. Z njim upošteva, da lahko sosednji slikovni elementi predstavljajo točke v prostoru, ki so si različno oddaljene.

\begin{equation}\label{eq:regularizacijski-del-pdflow}
E_R = \sum_\Omega \Psi\left( (\nabla w_x)^\top \vec{r} \right) + \Psi\left( (\nabla w_y)^\top \vec{r} \right) + \Psi\left( (\nabla \dot{Z})^\top \vec{r} \right)
\end{equation}

\begin{equation}\label{eq:faktor-prizora}
\vec{r} =
\begin{bmatrix}
\frac{1}{\sqrt{\frac{\partial X^2}{\partial x} + \frac{\partial Z^2}{\partial x}}} &
\frac{1}{\sqrt{\frac{\partial Y^2}{\partial y} + \frac{\partial Z^2}{\partial y}}}
\end{bmatrix}^\top
\end{equation}

Evaluacija algoritma PD-Flow je prikazana v tabeli~\ref{tab:pdflow}. V tabeli so zapisani še rezultati algoritma RGB-D flow, ki za cenilko $\Psi$ ravno tako uporablja normo $L_1$ o~\cite{jaimez2015primal}. Opazimo lahko, da se PD-Flow po metrikah bolje odnese. Največje izboljšanje vidimo pri času izvajanja algoritma.

\begin{table}[htb]
	\centering
    \begin{tabular}{l S[table-format=1.3] S[table-format=2.3] S[table-format=3.3] S[table-format=1.3]}
    \toprule
    \textbf{Algoritem} & \multicolumn{1}{c}{\textbf{NRMS-V}} & \multicolumn{1}{c}{\textbf{AAE}} & \multicolumn{1}{c}{\textbf{Čas izvajanja [s]}} & \multicolumn{1}{c}{\textbf{MAX-V [m]}} \\
    \midrule
    \textbf{PD-Flow} & \boldentry{1}{3}{0.068} & \boldentry{2}{3}{6.653} & \boldentry{3}{3}{7.150} & 0.111 \\
    RGB-D flow & 0.096 & 15.58 & 119.1 & 0.111 \\
    \bottomrule
    \end{tabular}
    \caption[Evaluacija algoritma PD-Flow]{Evaluacija algoritma PD-Flow in primerjava z algoritmom RGB-D flow, ki uporablja enako cenilko $\Psi$~\cite{jaimez2015primal}. Za metrike se uporabljata povprečna kotna napaka (AAE) in normaliziran koren srednje kvadratične napake magnitude hitrosti (NRMS-V), kjer se največja magnituda (MAX-V) uporablja za normalizacijo~\cite{jaimez2015primal}. Opazimo lahko, da se PD-Flow po metrikah bolje odnese. Največji izboljšanje vidimo pri času izvajanja algoritma. Odebeljene vrednosti predstavljajo najboljšo vrednost.}
    \label{tab:pdflow}
\end{table}
