\section{Merjenje energijske porabe}\label{sec:merjenje}
Zhang et al. \cite{zhang2004improving} je izpostavil, da je fizična aktivnost zelo kompleksna, še posebej če upoštevamo različen življenjski slog posameznika. Merjenje energijske porabe zato predstavlja velik metodološki izziv.

Kot smo razložili v poglavju \ref{sec:energija} je energijska poraba pravzaprav toplota, ki izhaja iz energijskih procesov mišičnih celic \cite{scott2005misconceptions}. Energijsko porabo zato lahko določimo z merjenjem toplotnih izgub med subjektom in kalorimetrom \cite{levine2005measurement}. Tovrstno merjenje imenujemo \textbf{direktna kalorimetrija}. Merilne naprave za direktno kalorimetrijo so izjemno drage, ki jih uporabljajo le visoko specializirani laboratoriji \cite{levine2005measurement}. Obstaja več tipov naprav, za vse pa je značilno, da so to komore, ki zagotavljajo toplotno ravnovesje. Odzivni časi so dokaj dolgi in lahko trajajo do \SI{30}{\min}, merilna napaka pa se giblje med \SI{1}{\%} - \SI{2}{\%} \cite{levine2005measurement}.

Toplota v človeškem telesu nastaja zaradi aerobnega ali anaerobnega metabolizma \cite{scott2005misconceptions}. Ker ima anaerobni metabolizem nizko kapaciteto in traja kratek čas \cite{sahlin1998energy}, se je v športu bolj uveljavilo merjenje aerobne kapacitete \cite{scott2005misconceptions,howley1995criteria}. Pri aerobnem metabolizmu se za produkcijo toplote porablja kisik, zato lahko energijsko porabo posredno merimo s porabo kisika (${VO}_2$) \cite{scott2005misconceptions}. Tako merjenje imenujemo \textbf{indirektna kalorimetrija} \cite{levine2005measurement}. Merilne naprave so glede na direktno kalorimetrijo cenejše in manj kompleksne. Večinoma gre za naprave z masko, ki mora biti fiksirana na nos in usta \cite{levine2005measurement}, zato niso primerne za široko uporabo ali izven-laboratorijske preiskave. Z merilnimi napakami pod \SI{3}{\%} in dokaj hitrimi odzivnimi časi prekašajo metode direktne kalorimetrije \cite{levine2005measurement}. 

Za potrebe terenskih preiskav se je razvila tretja skupina merilnih tehnik, t.i. \textbf{nekalorimetrične metode} \cite{levine2005measurement}. Energijska poraba nastaja zaradi gibanja telesa, zato se nekalorimetrične metode osredotočajo na opazovanje kinematike in ostalih fizioloških parametrov, ki sodelujejo pri fizičnih aktivnostih \cite{levine2005measurement}. Sem sodijo meritve srčnega utripa, elektromiografija, uporaba pedometrov in pospeškometrov ter brezkontaktne metode.

\subsection{Srčni utrip}
Pri zmerni fizični aktivnosti obstaja linearna povezava med srčnim utripom in porabo kisika \cite{keytel2005prediction}. Kar pa težko rečemo za odnos do energijske porabe, saj obstaja velika varianca med posamezniki \cite{levine2005measurement}. Ta je odvisna od fizioloških parametrov, kot so spol, višina, teža, ter telesna pripravljenost. Prav tako na pravilno estimacijo energijske porabe iz srčnega utripa vplivajo emocije in okoljske spremembe \cite{keytel2005prediction}. Srčni utrip, lahko zato uporabimo le v ozkem področju med \SI{90}{bpm} in \SI{150}{bpm}. Vendar še tu lahko dobimo razlike na intervalu $[-20~\% , 25~\%]$ glede na meritve indirektnih metod \cite{keytel2005prediction}. 

Ker je srčni utrip zelo slab posrednik za estimacijo energijske porabe, so raziskovalci predlagali modele, ki upoštevajo dodatne fiziološke parametre \cite{charlot2014improvement}. Najbolj pogosto citirana modela, ki se uporabljata tudi za široko populacijo sta Keytelova modela \cite{keytel2005prediction}. Pri prvem modelu \eqref{eq:keytel1} moramo za izračun energijske porabe $W_p$ (\si{\kcal.\min^{-1}}) poznati spol $s$ ($1$ moški, $0$ ženska), starost $st$ (leto), težo $m$ (\si{\kg}), srčni utrip $hr$ (\si{bpm}) in maksimalno porabo kiska merjenca $VO_{2max}$ (\si{\ml.\kg^{-1}.\min^{-1}}). Korelacijski koeficient $R$ tega modela glede na indirektno kalorimetrijo znaša $0.812$ \cite{charlot2014improvement}.

\begin{align} \label{eq:keytel1}
W_p = & -59.3954 + s \cdot (-36.3781 + 0.271 \cdot st + 0.394 \cdot m  \nonumber \\
& + 0.404 \cdot v + 0.634 \cdot hr ) + (1 - s) \nonumber \\
& \cdot (0.274 \cdot st + 0.103 \cdot m + 0.380 \cdot VO_{2max} + 0.450 \cdot hr)
\end{align}

Drugi Keytelov model \eqref{eq:keytel2} ne upošteva maksimalne porabe kisika $VO_{2max}$, ki nam pogosto manjka in je zato manj točen \cite{keytel2005prediction}. Njegov korelacijski koeficient $R$ znaša $0.632$ \cite{charlot2014improvement}.

\begin{align}\label{eq:keytel2}
 W_p = & s \cdot (-55.0969 + 0.6309 \cdot hr + 0.1988 \cdot m + 0.2017 \cdot st) \nonumber \\
 & + (1 - s) \cdot (-20.4022 + 0.4472 \cdot hr - 0.1263 \cdot m + 0.074 \cdot st)
\end{align}

Charlot et al. \cite{charlot2014improvement} je z uporabo drugačnih parametrov izboljšala rezultate glede na drugi Keytelov model. Model \eqref{eq:charlot} je tako dosegel korelacijski koeficient $R$ $0.657$. Pri tem modelu moramo za izračun energijske porabe $W_p$ (\si{\kcal.\hour^{-1}})  poznati srčni utrip $hr$ (\si{bpm}), višino $h$ (\si{\cm}), težo $m$ (\si{\kg}), spol $s$ ($1$ moški, $2$ ženski), srčni utrip v mirovanju $hr_r$ (\si{bpm}) in teoretični maksimalni srčni utrip $hr_{tmax}$ (\si{bpm}) \cite{charlot2014improvement}. Srčni utrip v mirovanju je definiran kot srednja vrednost srčnega utripa zadnjih \SI{2}{\min} \SI{5}{\min} mirovanja v ležečem položaju. Teoretični maksimalni srčni utrip lahko izračunamo na več različnih načinov. Najbolj pogosto uporabljena enačba za izračun je \eqref{eq:hrtmax1}, vendar pa obstajajo bolj natačni modeli, kot je enačba \eqref{eq:hrtmax2} \cite{miller1993predicting}. 

\begin{align}\label{eq:charlot}
W_p = & 171.62 + 6.87 \cdot hr + 3.99 \cdot h + 2.3 \cdot m \nonumber \\
& - 139.89 \cdot s - 4.26 \cdot hr_r - 4.87 \cdot hr_{tmax}
\end{align}

\begin{align}
	hr_{tmax} = & 220 - st \label{eq:hrtmax1}\\ 
    hr_{tmax} = & 217 - 0.85 \cdot st \label{eq:hrtmax2}
\end{align}

Slaba lastnost modela \eqref{eq:charlot} je, da energijsko porabo računamo na urni interval in ne na minutni kot je to običajno. Pri pretvorbi v minutni interval dobimo približek, saj s tem upoštevamo konstantno vrednost energijske porabe na intervalu $1$ ure. 

\subsection{Sensorji gibanja}
Za predikcijo energijske porabe iz opazovanja kinematike se večinoma uporabljajo pedometri in pospeškometri \cite{levine2005measurement}. Pedometri zaznajo premike z vsakim korakom, vendar pa imajo probleme z občutljivostjo. Ker  z njimi ne moremo določiti dolžine koraka so zelo slabi prediktorji in se za tovrstna merjenja ne uporabljajo \cite{levine2005measurement}.

Merjenje s pospeškometri je lahko dokaj natančno, saj je pospešek sorazmeren zunanjim silam in zato reflektira intenziteto gibanja \cite{yang2010review}. Pri tem moramo paziti, da uporabimo triosne in ne enoosnih pospeškometrov, saj ti ne dajejo zadovoljivih rezultatov \cite{levine2005measurement}. Pospeškometri se lahko uporabljajo tako za laboratorijske kot terenske raziskave \cite{yang2014sleep}, vendar pa kot pravi Zhang et al \cite{zhang2004improving} je njihova natančnost vprašljiva. Faktorji, ki vplivajo na njihovo natančnost so lokacija in način pritrditve na telo in zunanje vibracije \cite{yang2010review}. Priporočljivo je, da jih pritrdimo na spodnji del hrbta, saj bomo le tako zajeli večino premikov težišča telesa pri aktivnostih. za bolj natačne meritve bi morali pospeškometre pritrditi tudi na druge dele telesa, še posebej na okončine \cite{yang2010review}. To zmanjšuje njihovo praktičnost, saj omejujejo gibanje športnikov in s tem posredno vplivajo na rezultat. Pospeškometri pa imajo še eno slabo lastnost. Njihova natačnost močno upade kadar gibanje ni horizontalno s podlago, kar pomeni, da so neuporabni za hojo ali tek v hrib, plezanje, itd. \cite{yang2010review}. 

\subsection{Brezkontakne metode}
Zaradi omejitev kontaktnih senzorjev, slabih lastnosti srčnega utripa in drage indirektne kalorimetrije, ki se lahko uporablja samo v laboratoriju so raziskovalci razvili nekontaktne metode estimacije energijske porabe, kjer dominirajo metode analize video posnetkov \cite{botton2011energy,osgnach2010energy,silva2015assessing,peker2004framework,nathan2015estimating}.

Študije brezkontaktnega merjenja aktivnosti so redke, saj večina raziskovalcev uporablja kontaktne merilnike. Kljub temu so nekateri poskušali oceniti energijsko porabo z brezkontaktnimi senzorji gibanja, kot je Kinect \cite{nathan2015estimating}. 

Z estimacijo telesne aktivnosti na podlagi gibanja so se ukvarjali tudi v delih \cite{silva2015assessing,peker2004framework}, vendar pa je bila tu intenzivnost določena po subjektivni lestvici.

V delu \cite{osgnach2010energy} so avtorji pokazali, da lahko z video analizo v povezavi z metaboličnim modelom ocenimo energijsko zahtevnost igranja nogometa. Video posnetke so obdelali tako, da so sledili igralcem med celotno igro. Na podlagi sledenja so klasificirali gibanje v posamezne kategorije in iz fizioloških značilnosti izračunali porabo energije.

Podoben pristop k merjenju aktivnosti so predstavili v \cite{botton2011energy}. Bistvenim aktivnostim, ki se izvajajo v tenisu, so določili aerobno porabo energije in tako zgradili matematični model metabolizma. Iz video posnetkov so nato določili profil aktivnosti in tako posredno ocenili porabo energije.

Problem v zgornjih dveh pristopih je v modeliranju metaboličnega modela, ki je omejen na določeno vrsto aktivnosti ali športa. Prav tako študiji uporabljata analizo posnetkov, ki je bolj namenjena določitvi porabe energije po koncu igre in ne toliko sprotni ocenitvi tega pomembnega fiziološkega parametra.

Nobena od zgoraj opisanih metod ne izkorišča polja gibanja, ki najbolj kvalitetno opisuje intenzivnost telesne aktivnosti. Zato v tem delu predlagamo novo metodo estimacije intenzivnosti telesne aktivnosti na podlagi značilk iz video posnetkov. Dolgoročni cilj tovrstnega razvoja je metoda, ki bi se uporabila kot popolnoma nemoteč, brezkontakten merilni instrument.

Kljub specifični uporabi predlagane metode za merjenje energijske porabe, bi lahko tako metodo uporabili tudi za druge principe gibanja. Koncept merjenja s pomočjo optičnega toka nam omogoča, da opravljamo meritve z daljših razdalj, dokler nam optični sistem zagotavlja stabilno sliko. Tako smo se za prikaz posplošitve našega sistema odločili, da bomo preizkusili naš sistem kot detektor dihanja, saj dihanje, kot gibanje, ravno tako predstavlja vrsto telesne aktivnosti.