\section{Prispevek dela}
Nobena od brezkontaknih metod estimacije energijske porabe ne izkorišča polja gibanja, ki najbolj kvalitetno opisuje intenzivnost telesne aktivnosti. Zato v tem delu predlagamo novo metodo estimacije intenzivnosti telesne aktivnosti na podlagi značilk iz video posnetkov. Dolgoročni cilj tovrstnega razvoja je metoda, ki bi se uporabila kot popolnoma nemoteč, brezkontakten merilni instrument.

Kljub specifični uporabi predlagane metode za merjenje energijske porabe, bi lahko tako metodo uporabili tudi za druge principe gibanja. Koncept merjenja s pomočjo optičnega toka nam omogoča, da opravljamo meritve z daljših razdalj, dokler nam optični sistem zagotavlja stabilno sliko. Tako smo se za prikaz posplošitve našega sistema odločili, da bomo preizkusili naš sistem kot detektor dihanja, saj dihanje, kot gibanje, ravno tako predstavlja vrsto telesne aktivnosti. Detekcija dihanja je podrobneje predstavljena v poglavju \ref{sec:detekcija-dihanja}

V poglavju \ref{sec:metode}  predstavljamo metode, ki smo jih uporabili v postopku predikcije dihanja in predikcije energijske porabe. V poglavju \ref{sec:eksperimenti} so opisani eksperimenti in njihovi rezultati. Na koncu sledi še diskusija, kjer ovrednotimo rezultate. 
