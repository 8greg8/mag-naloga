\section{Prispevek dela}
Nobena od brezkontaktnih metod estimacije energijske porabe ne izkorišča polja gibanja. To je najbolj optimalna rešitev za sisteme računalniškega vida, ki merijo energijsko porabo, saj je povezana s kinematičnim gibanjem. Polja gibanja ne moremo direktno meriti, zato uporabljamo njegovo aproksimacijo, optični tok. Uporaba optičnega toka je lahko zagotovo bolj natančna od do sedaj predlaganih metod. Seveda tak pristop na podlagi natančnosti ne more nadomestiti indirektne kalorimetrije, lahko pa nadomesti široko uporabljene kontaktne senzorje. Metoda je v primerjavi s kontaknimi senzorji tudi bolj praktična, saj ti omejujejo gibanje in s tem posredno vplivajo na rezultat. 

Pri uporabi optičnega toka lahko veliko elementarnih problemov, kot so problem reže in paralaksa gibanja, rešimo z vpeljavo skrbno izbranih deskriptorjev. Tako lahko metodo uporabimo pri različnih modalitetah. Energijsko porabo lahko merimo iz različnih zornih kotov in z različnimi tipi kamer, kot so: RGB, bližnje-infrardeče (NIR) in globinske kamere (RGB-D). Z uporabo deskriptorjev lahko v postopek pridobivanja energijske porabe učinkovito integriramo regresijske modele s strojnim učenjem, na podlagi podpornih vektrojev in parametrično optimizacijo mrežnega iskanja. 

Vsekakor optični tok ni edini aproksimator polja gibanja. Koncept lahko razširimo na tridimenzionalni prostor z uporabo prostorskega toka. S slednjim lahko močno izboljšamo natančnost postopka, saj dobimo podatke v metričnih enotah. Tudi deskriptorje, ki jih uporabljamo za večjo robustnost optičnega toka, lahko razširimo na več dimenzij in tako obdržimo temeljni del postopka.

Kljub specifični uporabi predlagane metode za merjenje energijske porabe, bi lahko tako metodo uporabili tudi za druge principe gibanja. Koncept merjenja s pomočjo optičnega toka nam omogoča, da opravljamo meritve z daljših razdalj, dokler nam optični sistem zagotavlja stabilno sliko. Tako smo se za prikaz posplošitve našega sistema odločili, da bomo preizkusili naš sistem kot detektor dihanja, saj dihanje, enako kot gibanje, predstavlja vrsto telesne aktivnosti. Detekcija dihanja je podrobneje predstavljena v poglavju~\ref{sec:detekcija-dihanja}

V delu predstavljamo izčrpno študijo o pridobivanju energijske porabe iz podatkov 2D ali 3D slike, brez domnev ali detekcij postavitve skeleta. Preučili smo več modalitet vhodnih podatkov (RGB, NIR in t.i. čas preleta, angl. ``Time-of-flight''), različne pozicije kamer, različne tehnologije za pridobivanje videoposnetkov (IP kamere, vgrajena platforma Raspberry Pi, Microsoft Kinect za Windows V2) in različne kombinacije obdelovalnih elementov (HOOF in HAFA deskriptorji, sledenje, filtriranje, glajenje). Poskusi so bili opravljeni v laboratoriju in na terenu--med tekmami za squash.

V poglavju~\ref{sec:metode}  predstavljamo metode, ki smo jih uporabili v postopku predikcije dihanja in predikcije energijske porabe. V poglavju~\ref{sec:eksperimenti} opisujemo eksperimente in njihove rezultate. Na koncu sledi še diskusija, kjer ovrednotimo rezultate.

Manjši del študije, ki jo predstavljamo v tem delu, je bil objavljen na mednarodni konferenci~\cite{koporec2017observation}