\section{Detekcija dihanja}\label{sec:detekcija-dihanja}
Detekcija dihanja, je zelo pomembna v medicini, saj je dihanje eno izmed osnovnih življenjskih funkcij. Seveda obstaja že veliko literature in aplikacij na to temo~\cite{sathyanarayana2015vision}.

Velik poudarek pri spremljanju dihanja dajejo detekciji  \emph{sindroma spalne apneje}~\cite{sathyanarayana2015vision}. Gre za pogosto in resno zdravstveno stanje~\cite{wang2006vision}, kjer prihaja do krajših zastojev dihanja~\cite{flemons2002obstructive}. Zastoji dihanja se pojavljajo skozi celo noč in zmanjšujejo kvaliteto spanca. Pomanjkanje spanca vpliva na kvaliteto življenja in povečuje nagnjenost k zdravstvenim težavam~\cite{malhotra2002obstructive}. Apneja lahko povzroča depresijo in diabetes. Prav tako je povezana s kardiovaskularnimi obolenji~\cite{takemura2005respiratory}.

Obstajata dva tipa sindroma spalne apneje. Vzrok \emph{centralne spalne apneje} je okvara možganskega centra za nadziranje dihanja~\cite{javaheri2010central}. Možgani so nezmožni generiranja signalov za ritmično dihanje, kar povzroči pomanjkanje respiratornega gibanja prsnega koša in abdomna. \emph{Obstruktivno spalno apnejo} povzroči kolaps mehkega tkiva v grlu. Tkivo zapre dihalne poti, kar povzroči premikanje prsnega koša in abdomna v nasprotni smeri.

Trenutni klinični standard za diagnozo sindroma spalne apneje je \emph{polisomnograf} (PSG)~\cite{collop2007clinical}, kjer uporabljamo različne kontaktne senzorje za meritve različnih fizioloških parametrov~\cite{heinrich2015video}. Ker so senzorji pritrjeni na pacienta, ga zato motijo med spanjem. Kljub problemom kontaktnih senzorjev, so tovrstne meritve dokaj natančne z nizko stopnjo napak. Največji problem polisomnografske meritve je v tem, da je zelo draga in ni primerna za dolgotrajno opazovanje pacientov.

Bolj dostopna alternativa za detekcijo respiratornih motenj je \emph{pulzna oksimetrija}~\cite{netzer2001overnight}. Tu merimo ponavljajoče se fluktuacije v nasičenosti kisika v arterijah ($\mathrm{SpO}_{2}$)~\cite{levy1996accuracy}. Obstaja kar nekaj kazalnikov za predikcijo sindroma spalne apneje, ki so opisani v~\cite{netzer2001overnight, magalang2003prediction}. Najpogosteje se uporablja \SI{4}{\%} zmanjšanje nasičenosti glede na delovno točko signala. Na podlagi raznih raziskav~\cite{cooper1991value,magalang2003prediction,netzer2001overnight,levy1996accuracy} je natančnost oksimetrije podobna natačnosti PSG, zato se lahko uporablja kot alternativa za diagnozo sindroma spalne apneje.

Po drugi strani danes obstaja že kar nekaj brezkontaktnih metod na podlagi računalniškega vida, ki ne motijo spanca in posledično ne ogrožajo rezultatov. Večinoma se uporabljajo infrardeče kamere s sledenjem premikanja prsnega koša~\cite{sathyanarayana2015vision}. Nekateri so poskušali tudi z globinskimi kamerami~\cite{yang2014sleep}, kamerami s časom preleta~\cite{falie2009statistical} in razvojem namenskih senzorjev~\cite{takemura2005respiratory}.

Za sledenje premikanja prsnega koša se uporabljajo standardne metode za detekcijo gibanja. Razliko med slikami so uporabili v delu~\cite{nakai2000non}, optični tok pa v delu~\cite{nakajima2001development}. Ker je dihanje ciklično gibanje in je nagnjeno k okluzijam, avtorji v delu~\cite{wang2014unconstrained} zagovarjajo stališče, da te metode niso primerne za to problematiko. V našem preliminarnem delu~\cite{koporec2017observation} smo pokazali, da lahko z uporabo namenskih deskriptorjev naredimo bolj robusten optični tok, s katerim smo sposobni detektirati dihanje.