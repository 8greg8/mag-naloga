\section{Merjenje energijske porabe}\label{sec:merjenje}
Zaradi kompleksne narave fizične aktivnosti je merjenje energijske porabe velik metodološki izziv~\cite{zhang2004improving}. Zhang et al.~\cite{zhang2004improving} pri tem izpostavlja, da je pomemben del tega problema različen življenjski slog posameznika. 

%Kot smo razložili v poglavju~\ref{sec:energija} je energijska poraba pravzaprav toplota, ki izhaja iz energijskih procesov mišičnih celic~\cite{scott2005misconceptions}. 
Energijsko porabo lahko določimo z merjenjem toplotnih izgub med subjektom in kalorimetrom, saj se teoretično vsa mehanična energija v izoliranem sistemu pretvori v toploto~\cite{levine2005measurement}. Tovrstno merjenje imenujemo \emph{direktna kalorimetrija}~\cite{levine2005measurement}. Merilne naprave za direktno kalorimetrijo so izjemno drage, ki jih uporabljajo le visoko specializirani laboratoriji~\cite{levine2005measurement}. Obstaja več tipov naprav, za vse pa je značilno, da gre za komore, ki zagotavljajo toplotno ravnovesje. Odzivni časi so dokaj dolgi in lahko trajajo do \SI{30}{\min}, merilna napaka pa se giblje med \hbox{\SIrange{1}{2}{\%}}~\cite{levine2005measurement}.

Toplota v človeškem telesu nastaja zaradi aerobnega ali anaerobnega metabolizma~\cite{scott2005misconceptions}. Ker ima anaerobni metabolizem nizko kapaciteto in traja kratek čas~\cite{sahlin1998energy}, se je v športu bolj uveljavilo merjenje aerobne kapacitete~\cite{scott2005misconceptions,howley1995criteria}. Pri aerobnem metabolizmu se za produkcijo toplote porablja kisik, zato lahko energijsko porabo posredno merimo s porabo kisika (${VO}_2$)~\cite{scott2005misconceptions}. Tako merjenje imenujemo \emph{indirektna kalorimetrija}~\cite{levine2005measurement}. Merilne naprave so glede na direktno kalorimetrijo cenejše in manj kompleksne. Večinoma gre za naprave z masko, ki mora biti fiksirana na nos in usta~\cite{levine2005measurement}, zato niso primerne za široko uporabo ali izven-laboratorijske raziskave. Z merilnimi napakami pod \SI{3}{\%} in dokaj hitrimi odzivnimi časi prekašajo metode direktne kalorimetrije~\cite{levine2005measurement}.

Za potrebe terenskih raziskav se je razvila tretja skupina merilnih tehnik, t.i. \emph{nekalorimetrične metode}~\cite{levine2005measurement}. Energijska poraba nastaja zaradi gibanja telesa, zato se nekalorimetrične metode osredotočajo na opazovanje kinematike in ostalih fizioloških parametrov, ki sodelujejo pri fizičnih aktivnostih~\cite{levine2005measurement}. Sem sodijo meritve srčnega utripa, elektromiografija, uporaba pedometrov in pospeškometrov ter brezkontaktne metode.

\subsection{Srčni utrip}\label{sec:srcni-utrip}
Pri zmerni fizični aktivnosti obstaja linearna povezava med srčnim utripom in porabo kisika~\cite{keytel2005prediction}. Kar pa težko rečemo za odnos do energijske porabe, saj obstaja velika varianca med posamezniki~\cite{levine2005measurement}. Ta je odvisna od fizioloških parametrov, kot so spol, višina, teža in telesna pripravljenost. Prav tako na pravilno estimacijo energijske porabe iz srčnega utripa vplivajo emocije in okoljske spremembe~\cite{keytel2005prediction}. Srčni utrip lahko zato uporabimo le v ozkem področju med \SI{90}{bpm} in \SI{150}{bpm}. Vendar tudi tu lahko dobimo razlike na intervalu $[-20~\% , 25~\%]$ glede na meritve indirektnih metod~\cite{keytel2005prediction}. 

Ker je srčni utrip zelo slab posrednik za estimacijo energijske porabe, so raziskovalci predlagali modele, ki upoštevajo dodatne fiziološke parametre~\cite{charlot2014improvement}. Najbolj pogosto citirana modela, ki se uporabljata tudi za široko populacijo, sta \emph{Keytelova modela}~\cite{keytel2005prediction}. Pri prvem modelu~\eqref{eq:keytel1} moramo za izračun energijske porabe $W_p$ [\si{\kcal\per\min}] poznati spol $s$ ($1$ moški, $0$ ženska), starost $st$ [leto], težo $m$ [\si{\kg}], srčni utrip $hr$ [\si{bpm}] in maksimalno porabo kiska merjenca \vomax [\si{\ml\per\kg\per\min}]. Korelacijski koeficient $CORR$ tega modela glede na indirektno kalorimetrijo znaša \num{0.812}~\cite{charlot2014improvement}.

\begin{align} \label{eq:keytel1}
W_p = & \num{-59.3954} + s~(\num{-36.3781} + \num{0.271}~st + \num{0.394}~m  \nonumber \\
& + \num{0.404}~v + \num{0.634}~hr ) + (1 - s) \nonumber \\
&\cdot (\num{0.274}~st + \num{0.103}~m + \num{0.380}~VO_{2max} + \num{0.450}~hr)
\end{align}

Drugi Keytelov model~\eqref{eq:keytel2} ne upošteva maksimalne porabe kisika \vomax, ki nam pogosto manjka, in je zato manj točen~\cite{keytel2005prediction}. Njegov korelacijski koeficient \corr znaša \num{0.632}~\cite{charlot2014improvement}.

\begin{align}\label{eq:keytel2}
 W_p = & s~(\num{-55.0969} + \num{0.6309}~hr + \num{0.1988}~m + \num{0.2017}~st) \nonumber \\
 & + (1 - s) \cdot (\num{-20.4022} + \num{0.4472}~hr - \num{0.1263}~m + \num{0.074}~st)
\end{align}

Charlot et al.~\cite{charlot2014improvement} je z uporabo drugačnih parametrov izboljšala rezultate glede na drugi Keytelov model. Model~\eqref{eq:charlot} je tako dosegel korelacijski koeficient $\corr= \num{0.657}$. Pri tem modelu moramo za izračun energijske porabe $W_p$ [\si{\kcal\per\hour}]  poznati srčni utrip $hr$ [\si{bpm}], višino $h$ [\si{\cm}], težo $m$ [\si{\kg}], spol $s$ ($1$ moški, $2$ ženski), srčni utrip v mirovanju $hr_r$ [\si{bpm}] in teoretični maksimalni srčni utrip \hrtmax [\si{bpm}]~\cite{charlot2014improvement}. Srčni utrip v mirovanju je definiran kot srednja vrednost srčnega utripa zadnjih dveh minut pet minutnega mirovanja v ležečem položaju. Teoretični maksimalni srčni utrip lahko izračunamo na več različnih načinov. Najbolj pogosto uporabljena enačba za izračun je~\eqref{eq:hrtmax1}, vendar pa obstajajo bolj natačni modeli, kot je enačba~\eqref{eq:hrtmax2}~\cite{miller1993predicting}. 

\begin{align}\label{eq:charlot}
W_p = & \num{171.62} + \num{6.87}~hr + \num{3.99}~h + \num{2.3}~m \nonumber \\
& - \num{139.89}~s - \num{4.26}~hr_r - \num{4.87}~hr_{tmax}
\end{align}

\begin{align}
	hr_{tmax} = & \num{220} - st \label{eq:hrtmax1}\\ 
    hr_{tmax} = & 217 - \num{0.85}~st \label{eq:hrtmax2}
\end{align}

Slaba lastnost modela~\eqref{eq:charlot} je, da energijsko porabo računamo na urni interval in ne na minutnega, kot je običajno. Pri pretvorbi v minutni interval dobimo približek, saj s tem upoštevamo konstantno vrednost energijske porabe na intervalu $1$ ure. 





\subsection{Senzorji gibanja}\label{sec:senzorji-gibanja}

Za predikcijo energijske porabe iz opazovanja kinematike se večinoma uporabljajo pedometri in pospeškometri~\cite{levine2005measurement}. Pedometri zaznajo premike z vsakim korakom, vendar pa imajo probleme z občutljivostjo. Ker z njimi ne moremo določiti dolžine koraka, so zelo slabi prediktorji in se za tovrstna merjenja ne uporabljajo~\cite{levine2005measurement}.

Merjenje s pospeškometri je lahko dokaj natančno, saj je pospešek sorazmeren zunanjim silam in zato odraža intenziteto gibanja~\cite{yang2010review}. Pri tem moramo paziti, da uporabimo triosne in ne enoosnih pospeškometrov, saj ti ne dajejo zadovoljivih rezultatov~\cite{levine2005measurement}. Pospeškometri se lahko uporabljajo tako za laboratorijske kot tudi za terenske raziskave~\cite{yang2014sleep}, vendar pa, kot pravi Zhang et al~\cite{zhang2004improving}, je njihova natančnost vprašljiva. Faktorji, ki vplivajo na njihovo natančnost, so lokacija, način pritrditve na telo in zunanje vibracije~\cite{yang2010review}. Priporočljivo je, da jih pritrdimo na spodnji del hrbta, saj bomo le tako zajeli večino premikov težišča telesa pri aktivnostih. Za bolj natančne meritve bi morali pospeškometre pritrditi tudi na druge dele telesa, še posebej na okončine~\cite{yang2010review}. To zmanjšuje njihovo praktičnost, saj omejujejo gibanje športnikov in s tem posredno vplivajo na rezultat. Pospeškometri imajo še eno slabo lastnost. Njihova natančnost močno upade, kadar gibanje ni horizontalno s podlago, kar pomeni, da so neuporabni za hojo ali tek v hrib, plezanje, itd.~\cite{yang2010review}.

Za primer študije, ki uporablja kontaktne metode, lahko navedemo~\cite{gjoreski2015context}. V tem delu so avtorji določali energijsko porabo z regresijskimi modeli. Te so učili na večdimenzionalnih vektorjih značilk iz kontaktnih senzorjev.





\subsection{Brezkontaktne metode}

Zaradi omejitev kontaktnih senzorjev, slabih lastnosti srčnega utripa in drage indirektne kalorimetrije, ki se lahko uporablja samo v laboratoriju, so raziskovalci razvili brezkontaktne metode estimacije energijske porabe, kjer dominirajo metode analize video posnetkov~\cite{botton2011energy,osgnach2010energy,silva2015assessing,peker2004framework,nathan2015estimating}. 
