\section{Energija v človeškem telesu}\label{sec:energija}
Človeško telo lahko energijo porabi na tri različne načine:

\begin{itemize}
\item bazalna metabolična stopnja,
\item termični efekt hrane,
\item energijska poraba zaradi fizične aktivnosti~\cite{levine2005measurement}.
\end{itemize}

\emph{Bazalna metabolična stopnja} je uporabljena energija v času mirovanja (zjutraj, ko se zbudimo). Za povprečnega človeka predstavlja okoli \SI{60}{\%} dnevne porabe energije~\cite{levine2005measurement}. \emph{Termični efekt hrane} predstavlja porabo energije, ki je povezana s prebavo in shranjevanjem hrane. Predstavlja okoli \SI{10}{\%} dnevne porabe energije~\cite{levine2005measurement}. \emph{Energijska poraba zaradi fizične aktivnosti} predstavlja energijo, ki jo trošijo mišice. 

Da mišice lahko s svojimi kontrakcijami spravijo telo v pogon, potrebujejo njihove celice energijo, ki je shranjena v obliki molekulskih vezi adenozintrifosfata (ATP)~\cite{scott2005misconceptions}. Z razgradnjo ATP, celice dobijo potrebno energijo za kontrakcije, nato pa ponovno sintetizirajo ATP s pomočjo metabolizma iz amino kislin, ogljikovih hidratov in maščobnih kislin~\cite{scott2005misconceptions,patel2017aerobic}. ATP razgradnja in ponovna sinteza sta termodinamično ireverzibilni reakciji. ATP molekule predstavljajo energijsko kapaciteto, zato omejitev sinteze ATP dejansko povzroča omejitev energijske porabe in s tem zmogljivost sistema~\cite{sahlin1998energy}.

Za ponovno sintezo ATP molekul, lahko celice uporabljajo aerobni ali anaerobni metabolizem~\cite{scott2005misconceptions}. Pri \emph{aerobnem metabolizmu} formacija ATP poteka preko glikolize in Krebsovega cikla, kjer se porablja kisik $\mathrm{O}_2$. Diagram aerobnega metabolizma je predstavljen na sliki~\ref{fig:metabolism}. Ta se pojavlja večinoma pri ponavljajočih se ritmičnih gibih in manj intezivnih fizičnih aktivnostih, kot so kolesarjenje, daljši tek, ples, itd.~\cite{patel2017aerobic}. Ker se aerobni procesi uporabljajo za dolgotrajnejše dejavnosti, imajo visoko kapaciteto in nizko moč~\cite{sahlin1998energy}. Njihovo kapaciteto lahko merimo s pomočjo zmogljivosti kardio-respiratornega sistema~\cite{patel2017aerobic}. Večja dobava kisika bo posledično omogočila več aerobnega metabolizma in proizvajanja ATP molekul. Za kriterij aerobne kapacitete se je tako uveljavilo merjenje porabe kisika (${VO}_2$).

\emph{Anaerobni metabolizem} se pojavi, ko primanjkuje kisika, za produkcijo ATP molekul~\cite{patel2017aerobic}. Formacija ATP poteka preko glikolize in fermentacije, kar povzroča nizko raven ATP, sintezo mlečne kisline in akumulacijo laktata v krvi. Diagram anaerobnega metabolizma je predstavljen na sliki~\ref{fig:metabolism}. Mlečna kislina zakisa mišice, to pa vodi v mišično utrujenost~\cite{sahlin1998energy}. Anaerobni procesi se pojavijo ob intenzivnih aktivnostih, ki trajajo kratek čas (sprint, dviganje uteži...)~\cite{patel2017aerobic}. Imajo  veliko moč, vendar nizko kapaciteto~\cite{sahlin1998energy}.


\begin{figure}[!htbp]
\centering
\resizebox {\columnwidth} {!}{
\begin{tikzpicture}
  % LAYERS
  \pgfdeclarelayer{bg}
  \pgfsetlayers{bg,main}
  % LENGTHS
  \newlength{\outdistance}
  \setlength{\outdistance}{4mm}
  %STYLES
  
    
  %
  % Specification of nodes
  %
  \node (food) [input, text width=35mm] {Amino kisline, ogljikovi hidrati, maščobne kisline};
  \node (glycolysis) [block, below = of food] {Glikoliza};
  \node (o2) [input, below = of glycolysis] {O$_2$};
  
  \node (aerobic) [title, left = 2cm of glycolysis] {Aerobni metabolizem};
  \node (krebs) [block, below = of aerobic] {Krebsov cikel};
  \node (aerobic-atp) [output, below right = \outdistance of krebs] {veliko ATP};
  \node (h2o) [output, below left = \outdistance of krebs] {H$_2$O};
  \node (co2) [output, below = \outdistance of krebs] {CO$_2$};
  
  \node (anaerobic) [title, right = 2cm of glycolysis] {Anaerobni metabolizem}; 
  \node (ferment) [block, below = of anaerobic] {Fermentacija};
  \node (anaerobic-atp) [output, below right = \outdistance of ferment] {malo ATP};
  \node (acid) [output, below left = \outdistance of ferment] {Mlečna kis.};
  \node (lactat) [output, below = \outdistance of ferment] {Laktat};
  
  %
  % Specification of lines
  %
  \draw [arrow] (food.south) -- (glycolysis.north);
  \draw [arrow] (glycolysis.south west) -- (krebs.north);
  \draw [arrow] (glycolysis.south east) -- (ferment.north);
  \draw [arrow] (o2.west) -- (krebs.east);
  
  \draw [arrow] (krebs.south west) -- (h2o.north east);
  \draw [arrow] (krebs.south) -- (co2.north);
  \draw [arrow] (krebs.south east) -- (aerobic-atp.north west);
  
  \draw [arrow] (ferment.south west) -- (acid.north east);
  \draw [arrow] (ferment.south) -- (lactat.north);
  \draw [arrow] (ferment.south east) -- (anaerobic-atp.north west);
  
   %
  % Background
  %
  \begin{pgfonlayer}{bg}
  	\node (ltop) [above = 1mm of aerobic] {};
    \node (lleft) [left = 1mm of h2o] {};
  	\node (l0) [] at ( ltop -| lleft) {};
    \node (l1) [below right = 1mm of aerobic-atp] {};
    \path[background] (l0) rectangle (l1);
    
    \node (rtop) [above = 1mm of anaerobic] {};
    \node (rright) [right = 1mm of anaerobic-atp] {};
    \node (r0) [] at (rtop -| rright) {};
    \node (r1) [below left = 1mm of acid] {};
    \path [background] (r0) rectangle (r1);
  \end{pgfonlayer}
\end{tikzpicture}
}
 \caption[Diagram aerobnega in anaerobnega metabolizma]{Diagram aerobnega in anaerobnega metabolizma. Pri aerobnem metabolizmu formacija ATP poteka preko glikolize in Krebsovega cikla, kjer se porablja kisik~\cite{scott2005misconceptions}. Pri anaerobnem metabolizmu se ATP formirajo preko glikolize in fermentacije.}
 \label{fig:metabolism}
\end{figure}
