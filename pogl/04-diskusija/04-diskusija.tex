\chapter{Diskusija}\label{sec:diskusija}
V tem delu smo raziskali novo brezkontaktno metodo za estimacijo fizioloških parametrov iz gibanja. Za določitev parametrov smo uporabljali algoritme optičnega in prostorskega toka, ki smo jih kombinirali s kombinacijami HOOF in HAFA deskriptorjev. S tem smo zagotovili robustnost algoritmov. Za predikcijo energijske porabe in srčnega utripa smo uporabili SVM regresijo, pri tem pa smo razvili \nurbf postopek mrežnega iskanja za optimizacijo. Za izločevanje šuma iz ozadja in gibanja neopazovanih objektov smo uporabljali sledilnike ter Kalmanov in Gaussov filter.

Rezultati kažejo na ta to, da so izbrani fiziološki parametri \emph{observabilni}. Predlagano metodo lahko uporabimo z različnimi sistemi za vizualno zaznavanje z različnih zornih kotov. Boljše rezultate lahko dobimo z uporabo posnetkov iz več zornih kotov. Elementarni modeli iz prve faze eksperimentov niso primerni za terensko uporabo. Pri učenju modelov moramo biti pozorni na akumulacijo utrujenosti. Za laboratorijske preiskave je bolje, če uporabimo metode z optičnim tokom. Na terenu dobimo najboljše rezultate z uporabo prostorskega toka.

Za dela \cite{peker2004framework,silva2015assessing,osgnach2010energy} rezultatov ne moremo primerjati, ker v njih uporabljajo subjektivne mere. Prav tako ne vsebujejo nobene primerjave z uveljavljeno referenco indirektne kalorimetrije.

če primerjamo delo \cite{botton2011energy} z našimi končnimi generaliziranimi modeli za laboratorijske eksperimente faze 2, dobimo boljše rezultate. V delu \cite{botton2011energy} je korelacijski koeficient $CORR=0.93$. Naš znaša $CORR=0.97$. Pri primerjavi z generaliziranimi modeli za terenske preiskave faze 2 dobimo slabše rezultate, saj korelacijski koeficient znaša $CORR=0.74$.

V \cite{nathan2015estimating} je konkordančni korelacijski koeficient $CCC=0.879$. Njihova RMS napaka je $RMSE=8.384$. Za naše najboljše laboratorijske eksperimente faze 2 dobimo $CCC=0.896$ in $RMSE=1.785$. Za najboljše terenske modele druge faze so metrike  $CCC=0.641$ in $RMSE=2.296$. Za obe vrsti eksperimentov lahko zaključimo, da dobimo boljše rezultate. Pri laboratorijskih sta si $CCC$ zelo podobna, vendar mi proizvedemo veliko manjšo RMS napako. Pri terenskih testih je $CCC$ slabši, vendar pa je metrika $RMSE$ toliko boljša.

Avtorji so v \cite{gjoreski2015context} uporabili podoben pristop, le da so namesto brezkontaktnih uporabili kontaktne senzorje. Z njihovim MCE pristopom so dobili $RMSE=1.192$ za aktivnosti teka. V delu navajajo tudi metriko za BodyMedia senzor. Gre za trenutno najboljši komercialno dosegljiv kontaktni senzor za estimacijo energijske porabe. Njegova metrika je znašala $RMSE=2.458$ za aktivnosti teka. Glede na naše laboratorijske rezultate iz zgornjega odstavka lahko zaključimo, da dobimo podobne rezultate kot MCE metoda. Glede na najboljši kontaktni senzor dobimo boljše rezultate.

Kljub obetavnim rezultatom, metoda vsebuje še kar nekaj problemov, ki jih moramo rešiti v prihodnje. Nimamo modela z upoštevanjem časovne dinamike. Globinske slike Kinect senzorja vsebujejo veliko šuma, ki bi ga mogli pred procesiranjem čim bolj odpraviti. Združevanje slik iz dveh Kinect kamer ni idealno. Potrebovali bi bolj avtomatično metodo. V postopku bi lahko dodali nove najboljše sledilnike. S tem bi pridobili še večjo natančnost sistema. Modele faze 2 bi morali dodatno optimizirati, da bi dobili manjši šum na izhodu.