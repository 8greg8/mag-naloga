\chapter{Diskusija}\label{sec:diskusija}
V tem delu smo raziskali novo brezkontaktno metodo za estimacijo fizioloških parametrov iz gibanja. Za določitev parametrov smo uporabljali algoritme optičnega in prostorskega toka, ki smo jih kombinirali s kombinacijami HOOF in HAFA deskriptorji. S tem smo zagotovili robustnost algoritmov. Za predikcijo energijske porabe in srčnega utripa smo uporabili SVM regresijo, pri tem pa smo razvili \nurbf postopek mrežnega iskanja za optimizacijo. Za izločevanje šuma iz ozadja in gibanja neopazovanih objektov smo uporabljali sledilnike ter Kalmanov in Gaussov filter.

Rezultati kažejo na to, da so izbrani fiziološki parametri \emph{observabilni}. Predlagano metodo lahko uporabimo z različnimi sistemi za vizualno zaznavanje z različnih zornih kotov. Boljše rezultate lahko dobimo z uporabo posnetkov iz več zornih kotov. Elementarni modeli iz prve faze eksperimentov niso primerni za terensko uporabo. Pri učenju modelov moramo biti pozorni na akumulacijo utrujenosti. Za laboratorijske preiskave je bolje, če uporabimo metode z optičnim tokom. Na terenu dobimo najboljše rezultate z uporabo prostorskega toka.

Za dela~\cite{peker2004framework,silva2015assessing,osgnach2010energy} rezultatov ne moremo primerjati, ker v njih uporabljajo subjektivne mere. Prav tako ne vsebujejo nobene primerjave z uveljavljeno referenco indirektne kalorimetrije.

Če primerjamo delo~\cite{botton2011energy} z našimi končnimi generaliziranimi modeli s prostorskim tokom faze 2, dobimo boljše rezultate. V delu~\cite{botton2011energy} je korelacijski koeficient $\corr=\num{0.93}$. Naš znaša $\corr=\num{0.995}$ za laboratorijske in $\corr=\num{0.999}$ za terenske teste. 

V~\cite{nathan2015estimating} je konkordančni korelacijski koeficient $CCC=\num{0.879}$, napaka pa znaša $\rmse=\SI{2.004}{\kcal}$. Za naše najboljše laboratorijske modele s prostorskim tokom faze 2 dobimo $CCC=\num{0.989}$ in $\rmse=\SI{9.870}{\kcal}$. Za najboljše terenske modele s prostorskim tokom faze 2 so metrike  $CCC=\num{0.983}$ in $\rmse=\SI{4.234}{\kcal}$. V obeh primerih dobimo boljše korelacije, vendar pa so napake dosti večje. Napako bi lahko izboljšali z večanjem števila podpornih vektorjev in dodatnim glajenjem izhodnih signalov.

Avtorji so v~\cite{gjoreski2015context} uporabili podoben pristop našemu, le da so namesto brezkontaktnih uporabili kontaktne senzorje. Z njihovim MCE pristopom so dobili $\rmse=\SI{1.192}{MET}$ za aktivnosti teka. V delu navajajo tudi metriko za BodyMedia senzor. Gre za trenutno najboljši komercialno dosegljiv kontaktni senzor za estimacijo energijske porabe. Njegova metrika je znašala $\rmse=\SI{2.458}{MET}$ za aktivnosti teka. V našem primeru dobimo $\rmse=\SI{8.111}{MET}$ za laboratorijske modele s prostorskim tokom faze 2 in $\rmse=\SI{4.098}{MET}$ za terenske modele s prostorskim tokom faze 2. Rezultati so po izbranih metrikah dokaj slabi. 

Kljub specifični uporabi predlagane metode za merjenje energijske porabe, smo pokazali da lahko metodo uporabimo tudi za detekcijo dihanja. Natančnost našega detektorja je \SI{85}{\%}. Za primerjavo so avtorji v \cite{nakajima2001development} dobili \SI{84.2}{\%}. Zaključimo lahko, da je naša metoda z rahlimi modifikacijami primerna tudi za širšo uporabo.

Kljub obetavnim rezultatom metoda vsebuje še kar nekaj problemov, ki jih moramo rešiti v prihodnje. Nimamo modela z upoštevanjem časovne dinamike. Globinske slike Kinect senzorja vsebujejo veliko šuma, ki bi ga mogli pred procesiranjem čimbolj odpraviti. Združevanje slik iz dveh Kinect kamer ni idealno. Potrebovali bi bolj avtomatično metodo. V postopku bi lahko dodali nove najboljše sledilnike. S tem bi pridobili še večjo natančnost sistema. Modele faze 2 bi morali dodatno optimizirati, da bi dobili manjši šum na izhodu.

Nenazadnje pa velja omeniti še problematiko razlike med merjenjem \emph{obremenitve} in \emph{napora}. Zaradi praktičnih omejitev je edina široko razširjena referenčna meritev na področju športa še vedno posredna kalorimetrija, vendar pa ta meri dejanski odziv telesa merjenca na zunanjo obremenitev. Ta odziv se lahko spreminja glede na utrujenost, kot smo ugotovili tudi pri naših eksperimentih. Naša metoda pa dejansko meri obremenitev samo, kot je objektivno vidna iz gibanja merjenca, kar je pomemben podatek za športno treniranje, ne moremo pa pričakovati da se bo popolnoma ujemal s podatki, ki jih zajame posredna kalorimetrija. V vsakem primeru pa naša metoda omogoča bistveno večjo časovno ločljivost od kalorimetričnih metod, kar je tudi lahko vir razlike med podatki, pridobljenimi z našo metodo, in kalorimetričnimi metodami. V prid tej tezi govori tudi relativno visoka natančnost merjenja \emph{skupne obremenitve} glede na rezultate metrik, s katerimi smo ocenjevali časovni potek meritve skozi čas.