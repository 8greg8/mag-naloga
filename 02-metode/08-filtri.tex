\section{Filtri}\label{sec:filtri}
\% smith
digitalni filtri so zelo pomemnbi pri digitalnem procesiranju signalov. filtre uporabljamo za dve stvari: signal separation and signam restoration. Prva se uporablja ko imamo signal ki je kontaminiran z interferenco, šumom ali drugimi signali. 

signal restoration se uporablja ko je bil signal distorted in some way recimo debluring of an image.

vsak filter im a impulze resonse step response in frekvenčni odziv. filter implementiramo tako da uporabmo konvolucijo z vhodnim signalom. 

moving average filters
optimalen za reducing random noise while retaining a sharp step response. najboljši za signale v časovnem prostoru.  -> gaussov filter -> frekvenčni odziv ravno tako gaussian  

zero phase filter
je karakterziran z impulznim odzivom ki je simetričen okoli ničelnega vzorca. oblika ni pomembna le da so negativni vzorci zrcalna slika pozitivnih vzorcev. Faza bo v frekvenčnem prostoru 0. Ker imamo negativne indekse ga je težko uporabljati. 

forward and reverse
vsak rekurzivni filter lahko s tako tehniko pretvorimo v zero phase filter. signal filtriramo normalno potem še reverzno in rezultata spojimo skupaj. 







\subsection{Kalmanov filter}\label{sec:kalmanov-filter}
\% trucco
system model
fizični sistem je modeliran z vektorjem stanj x in z enačbami sistemski model. stanje je odvisno od časa. sistemski model je vektorska enačba ki govori o tem kako se model razvija skozi čas. diskretni čas: tk = t0 + k delta t. delta t mora biti dovolj majhen da zajamemo dinamiko sistema. torej da se ne spremeni za veliko med koraki. zeta je vektor kjer modeliramo sistemski šum. matriko prehanja stanj 

measurement model
predvidevamo da ob vsakem časovnem trenutku dobimo meritev stanja ki je šumna. 
kjer je H merilna matrika in ni predstavlja šum. 

oba šuma sta bela brez srednje vrednosti gaussova procesa s kovariančnimi matrikami. 

alogritem
imamo še kovariančno matriko stanja P, gain matriko K 


Modeli so dali pošumljen izhod, zato smo implementirali še Kalmanov filter \cite{forsyth2002computer}. V prostoru stanj je predstavljen z enačbo \eqref{eq:kalman-model}, kjer je $x$ stanje hitrosti $v$ in pospeška $a$ \eqref{eq:stanje}, $A$ matrika prehajanja stanj \eqref{eq:a}, $G$ matrika šuma \eqref{eq:g}, s katero modeliramo neznane vhodne parametre hitrosti $v_n$ in pospeška $a_n$ v vektorju $u$ \eqref{eq:u}. Začetna hitrost in pospešek sta bila $0$. Varianca procesnega šuma je za vse modele znašala \num{0.04}, varianca šuma modela pa \num{456.13}. Zaradi neznanih začetnih vrednosti smo za kovariančno matriko predikcije uporabili varianco \num{456.13}.

\begin{subequations}
	\begin{align}
		x(k+1) &= A x(k) + G u(k) \label{eq:kalman-model} \\ 
        x(k) &= \begin{bmatrix}
					v(k) & a(k)
				\end{bmatrix}^\top \label{eq:stanje} \\
        A &= \begin{bmatrix}
				1 & 1 \\
                0 & 1
			\end{bmatrix} \label{eq:a} \\
       G &= \begin{bmatrix}
				1 & 0
			\end{bmatrix}^\top \label{eq:g} \\
       u(k) &= \begin{bmatrix}
					v_{n}(k) & a_n(k)
				\end{bmatrix}^\top \label{eq:u}
	\end{align}	
\end{subequations}

\subsection{Gaussov filter}\label{sec:gaussov-filter}



\subsection{Optimizacija Gaussovega jedra}


