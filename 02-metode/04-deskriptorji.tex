\section{Deskriptorji}
% Težave optičnega toka in prostorskega toka.
% Kako bi te težave rešili z deskriptorji
% Našli specifične deskriptorje
Žal so klasične metode optičnega toka občutljive na šum, diskontinuitete gibanja ter spremembe v osvetljenosti objekta~\cite{brox2011large}, pri novejših pa še vedno obstaja problem pravilne ocene amplitude gibanja zaradi pojava paralakse~\cite{xu2012scale} -- objekti, ki so bolj oddaljeni od kamere imajo manjšo jakost optičnega toka. Zato smo se odločili, da gibanje iz video posnetkov predstavimo s histogrami orientiranega optičnega toka (HOOF) \cite{chaudhry2009histograms}, ki nekoliko izboljšajo robustnost.


\subsection{HOOF}
% Teorija teh deskriptorjev
% Kako smo jih mi uporabili
% Težave teh deskriptorjev
% Dodali nove deskriptorje

\subsection{HAAF}
% Opis dodatnih deskriptorjev
% Kako uporabili nove deskriptorje

\subsection{kalibracija velikosti glede na diagonalo}
% Zakaj naj bi bila ta rešitev dobra
% teorija tovrstne kalibracije
% Kako s to kalibracijo nismo popravili stvari
% In da smo ugotovili da obstaja še prostorski tok, ki bi nam rešil težave