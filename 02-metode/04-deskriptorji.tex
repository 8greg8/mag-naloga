\section{Deskriptorji}
% Težave optičnega toka in prostorskega toka.
% Kako bi te težave rešili z deskriptorji
% Našli specifične deskriptorje
Klasične metode optičnega toka so občutljive na šum, diskontinuitete gibanja ter spremembe v osvetljenosti objekta~\cite{brox2011large}, pri novejših pa še vedno obstaja problem pravilne ocene amplitude gibanja zaradi \textbf{pojava paralakse}~\cite{xu2012scale} -- objekti, ki so bolj oddaljeni od kamere imajo manjšo jakost optičnega toka. 

Ker je prostorski tok projekcija optičnega toka v prostor ima podobne probleme kot optični tok \cite{yan2016scene}. Večja natančnost algoritma zahteva večjo \textbf{računsko zahtevnost}, kar vodi v manjšo učinkovitost. \textbf{Okluzija}, ki se lahko pogostokrat pojavi, krši konsistentnost podatkov skozi čas, in lahko vodi v napačno določitev korespondec \cite{yan2016scene}. Pri \textbf{hitrem gibanju} večina algoritmov ne deluje, saj temeljijo na predpostavki kratkih premikov na časovno enoto. Zaradi \textbf{sprememb osvetlitve} prizora postane estimacija prostorskega toka neuporabna \cite{yan2016scene}. Prav tako lahko pride do problemov, ko imamo \textbf{pomanjkanje teksture}, saj težje izračunamo gradient.

Surova optični in prostorski tok zaradi vrste problemov nista primerna za opis gibanja, čeprav predstavljata najbolj naravno metodo estimacije energijske porabe. Tudi če zagotovimo idealno okolje (kontinuiteta gibanja, konstantna osvetljenost, počasno gibanje in dobra tekstura) imamo še vedno problem šuma zaradi CCD senzorja \cite{wedel2011stereo}. Ravno tako se ne moremo znebiti paralakse ali zagotoviti neodvisnosti od smeri po $X$ osi koordinatnega sistema kamere \cite{chaudhry2009histograms}. Pri tem moramo opozoriti še na dejstvo, da se število slikovnih elementov, ki predstavljajo merjenca, spreminja skozi čas. Vsa dejstva stremijo k temu, da moramo za pravilno merjenje energijske porabe uporabiti deskriptorje, ki izboljšajo robustnost optičnega in prostorskega toka \cite{chaudhry2009histograms}. 

\subsection{Histogrami orientiranega optičnega toka}
% Teorija teh deskriptorjev
% Kako smo jih mi uporabili
% Težave teh deskriptorjev
% Dodali nove deskriptorje
Ko se človek premika se optični tok temporalno spreminja. Lahko rečemo, da se spreminja karakteristični profil optičnega toka \cite{chaudhry2009histograms}. Prva ideja za deskriptor bi bila distribucija optičnega toka. Ker pa se profil spreminja zaradi paralakse, potrebujemo deskriptor, ki je invarianten na skalo in smer gibanja \cite{chaudhry2009histograms}.

Chaudhry et al. \cite{chaudhry2009histograms} predlaga uporabo histogramov orientiranega optičnega toka (HOOF), kjer vsak vektor optičnega toka zložimo v stolpec, glede na njegov kot in ga utežimo z njegovo velikostjo.

Vektorju optičnega toka $\vec{w} = [x~y]^\top$ določimo smer \eqref{eq:smer} in amplitudo \eqref{eq:amplituda} \cite{chaudhry2009histograms}. Interval smeri $\Theta$ je določen z \eqref{eq:interval}. 

\begin{align}
	\Theta = & \tan^{-1}\left( \frac{y}{x} \right) \label{eq:smer} \\
    \| \vec{w} \| = & \sqrt{x^2 + y^2} \label{eq:amplituda}
\end{align}

\begin{equation}\label{eq:interval}
	-\frac{\pi}{2} + \pi \frac{b - 1}{B} \leq \Theta < - \frac{\pi}{2} + \pi \frac{b}{B}
\end{equation}

Interval smeri \eqref{eq:interval} pomeni, da vektorju $\vec{w}$ določimo stolpec $b$, za katerega velja $1 \leq b \leq B$, pri čemer je $B$ celotno število stolpcev histograma, na podlagi smeri $\Theta$ \cite{chaudhry2009histograms}. Pri tem moramo za smer $\Theta$ upoštevati najmanjši predznačen kot med vektorjem $\vec{w}$ in koordinatno osjo $x$ koordinatnega sistema slikovne ravnine $\mathit{\Omega}$. Z drugimi besedami, upoštevamo samo kote na intervalu \eqref{eq:interval-kot}, kote na intervalu \eqref{eq:interval-kot2} pa preslikamo na interval \eqref{eq:interval-kot}. Interval \eqref{eq:interval-kot} razdelimo na $B$ podintervalov, ki predstavljajo stolpce histograma. 

\begin{equation}\label{eq:interval-kot}
	\left[-\frac{\pi}{2}, \frac{\pi}{2}\right]
\end{equation}

\begin{equation}\label{eq:interval-kot2}
	\left(\frac{\pi}{2},\frac{3\pi}{2}\right)
\end{equation}

Vsak vektor $\vec{w}$, ki leži v podintervalu ali stolpcu $b$ bo prispeval svojo velikost $\|\vec{w} \|$ k njegovi vsoti \cite{chaudhry2009histograms}. Dobljeni histogram še normaliziramo, tako da je njegova vsota enaka $1$.

Preslikava intervala \eqref{eq:interval-kot2} v interval \eqref{eq:interval-kot} omogoča neodvisnost histograma od leve ali desne smeri gibanja \cite{chaudhry2009histograms}. Če se subjekt premika v levo ali desno, bo histogram enak. Z normalizacijo histograma dobimo invariantnost na skalo \cite{chaudhry2009histograms}. Če se subjekt premika daleč ali blizu kamere, bo histogram enak. Ker je vsak prispevek vektorja sorazmeren njegovi amplitudi, šumni vektorji nimajo vpliva na obliko histograma \cite{chaudhry2009histograms}. Posledično lahko določimo histogram za celotno sliko in zato ne potrebujemo segmentacije ali subtrakcije gibajoče osebe iz ozadja. 

Edini parameter, ki ga moramo določiti za HOOF značilke je število stolpcev histograma $B$. Chaudry et al \cite{chaudhry2009histograms} pravi, da moramo za dobro delovanje določiti najmanj $30$ stolpcev. Za potrebe naše metode smo število stolpcev podvojili in tako uporabili empirično določen $B=60$. S takim številom smo zagotovili dobro delovanje glede na minimalno vrednost, še vseeno pa ne gre za tako veliko število, ko bi do izraza prišle amplitude šumnih vektorjev.



\subsection{HAAF}
% Opis dodatnih deskriptorjev
% Kako uporabili nove deskriptorje

Following results in the realistic environment (a squash court), we decided to augment HOOF descriptor with the histogram of the absolute flow amplitudes, which significantly improved observed correlation between reference measurement and predicted values. In our implementation, we use $N_{HOOF}=60$ segment HOOF feature vector, roughly representing 60 directions. This was later augmented with $N_{ampl}=60$ amplitude bins, roughly corresponding to flow amplitudes in 0.5-60 pixel range, yielding 120-dimensional descriptor per each person and frame. Amplitude cut-off at 0.5 pixel was employed to get rid of noise in absence of motion.


\subsection{kalibracija velikosti glede na diagonalo}
% Zakaj naj bi bila ta rešitev dobra
% teorija tovrstne kalibracije
% Kako s to kalibracijo nismo popravili stvari
% In da smo ugotovili da obstaja še prostorski tok, ki bi nam rešil težave
a