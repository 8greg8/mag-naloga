\chapter{Metode}\label{sec:metode}
\section{Model gibanja}\label{sec:model-gibanja}
% Ponovi teorijo energijske porabe in kako pridemo do gibanja
% Dokazi, da lahko gibanje opazujemo s kamerami
% Teorija, da je najboljši približek optični tok
Da mišice lahko s svojimi kontrakcijami spravijo telo v pogon, potrebujejo energijo, pri tem pa se del energije pretvarja v toploto, ki ji pravimo energijska poraba \cite{scott2005misconceptions}. Energijska poraba nastaja zaradi gibanja telesa, zato jo lahko določimo z opazovanjem kinematike \cite{levine2005measurement}. Zaradi omejitev kontaktnih senzorjev in slabih lastnosti srčnega utripa so v študijah, ki so opisane v poglavju \ref{sec:podobna-dela} dokazali, da lahko z opazovanjem kinematike z računalniškim vidom določimo energijsko porabo. 

Naj bo delec z maso $m$ predstavljen kot točka prizora $\vec{P} = [X~Y~Z]^\top$ , kjer $X$, $Y$ in $Z$ predstavljajo koordinate na posameznih oseh v koordinatnem sistemu kamere \cite{trucco1998introductory}. Gibanje delca $\vec{P}$ lahko predstavimo z vektorjem hitrosti $\vec{V} = [V_X~V_Y~V_Z]^\top$, kjer so $V_X$, $V_Y$ in $V_Z$ hitrosti glede na osi. Kadar imamo v prostoru več masnih delcev, množico vektorjev hitrosti imenujemo \textbf{polje hitrosti} (angl. Velocity Field) \cite{trucco1998introductory}. 

Relativno gibanje delca $P$ glede na koordinatno izhodišče kamere $O$ lahko opišemo kot:

\begin{equation}
	\vec{V} = -\vec{T}-\omega\times\vec{P},
\end{equation}

kjer je $\vec{T}$ translatorna hitrost in $\omega$ kotna hitrost \cite{trucco1998introductory}. Po komponentah lahko gibanje opišemo z enačbo \eqref{eq:gibanje}

\begin{equation} \label{eq:gibanje}
	\begin{bmatrix}
	V_x \\ V_Y \\ V_Z
	\end{bmatrix}
    =
    \begin{bmatrix}
    - T_X - \omega_Y Z + \omega_Z Y \\
    - T_Y - \omega_Z X + \omega_X Z \\
    - T_Z - \omega_X Y + \omega_Y X
    \end{bmatrix}.
\end{equation}

Gibanje telesa v prostoru torej lahko opišemo s poljem hitrosti gibanja.

%Pri najbolj splošni obliki detekcije gibanja iz video posnetkov, moramo izračunati gibanje posameznega slikovnega elementa \cite{szeliski2010computer}, tako imenovani \emph{optični tok}, ki pod določenimi predpostavkami predstavlja dober približek gibanja objekta na sliki. Žal so klasične metode optičnega toka občutljive na šum, diskontinuitete gibanja ter spremembe v osvetljenosti objekta~\cite{brox2011large}, pri novejših pa še vedno obstaja problem pravilne ocene amplitude gibanja zaradi pojava paralakse~\cite{xu2012scale} -- objekti, ki so bolj oddaljeni od kamere imajo manjšo jakost optičnega toka. Zato smo se odločili, da gibanje iz video posnetkov predstavimo s histogrami orientiranega optičnega toka (HOOF) \cite{chaudhry2009histograms}, ki nekoliko izboljšajo robustnost. Povezavo gibanja športnika z značilkami HOOF prikazuje slika, kot sledi.  

%V idealnem primeru lahko gibanje delca z maso $m$ razčlenimo na komponente hitrosti v vertikalni in horizontalni smeri. Gibanje v smeri $\vec{v}_{y+}$ predstavlja največji napor, saj mora mišica opravljati delo, ki kljubuje sili gravitacije. Nasprotno je lahko gibanje v smeri $\vec{v}_{y-}$ relativno nenaporno. Gibanje v smereh $\vec{v}_{x+}$ in $\vec{v}_{x-}$ je po vloženi energiji nekje vmes. Dela, ki pripada gibanju v poljubni smeri, ne znamo točno oceniti, s pomočjo značilk HOOF pa se ga lahko \emph{naučimo}.


\section{Optični tok}
% Teorija optičnega toka
% vrste optičnega toka
% Kateri tip smo mi uporabili
% Zakaj smo tega uporabili
% Kako smo ga uporabili
% Problemi naše predpostavke
% Rešitev s kalibracijo velikosti
Za namen razlage upoštevamo perpektivni model kamere, kjer je \cite{trucco1998introductory}:

\begin{itemize}
\item \textbf{optična os} enaka $Z$ osi koordinatnega sistema kamere.
\item \textbf{Gibajoči masni delec} je predstavljen enako kot v poglavju \ref{sec:model-gibanja}.
\item \textbf{Osvetlitev} se ne spreminja.
\end{itemize}

Sliko delca $\vec{p} = [x~y]^\top$ na slikovni ravnini, kjer sta $x$ in $y$ slikovni koordinati, lahko predstavimo z enačbo

\begin{equation}\label{eq:slika-delca}
	\vec{p} = f \frac{\vec{P}}{Z},
\end{equation}

kjer je $f$ goriščna razdalja kamere \cite{trucco1998introductory}. S časovnim odvodom enačbe \eqref{eq:slika-delca}, dobimo hitrost delca na slikovni ravnini:

\begin{equation}\label{eq:hitrost-slike-delca}
	\vec{v} = f \frac{Z\vec{V}-V_Z\vec{P}}{Z^2}.
\end{equation}

Razširjena oblika enačbe \eqref{eq:hitrost-slike-delca}, kjer upoštevamo \eqref{eq:gibanje}, je zapisana z enačbo \eqref{eq:hitrost-slike-delca-raz} \cite{trucco1998introductory}. Prvi člen v posamezni enačbi predstavlja \textbf{translatorni del}, ostali členi pa sodijo v \textbf{rotacijski del}.

\begin{equation}\label{eq:hitrost-slike-delca-raz}
\begin{aligned}
	v_x = & \frac{T_Z x - T_X f}{Z} - \omega_Y f + \omega_Z y + \frac{\omega_X x y}{f} - \frac{\omega_Y x^2}{f} \\
    v_y = & \frac{T_Z y - T_Y f}{Z} - \omega_X f + \omega_Z x + \frac{\omega_Y x y}{f} - \frac{\omega_X y^2}{f}
\end{aligned}
\end{equation}

Kadar imamo na slikovni ravnini več slik delcev, množico vektorjev hitrosti $\vec{v}$ imenujemo \textbf{polje gibanja} (angl. Motion Field) \cite{trucco1998introductory}. Polje gibanja lahko razumemo kot projekcijo polja hitrosti na slikovno ravnino, zato ta predstavlja idealno rekonstrukcijo gibanja. V praksi do polja gibanja ne moremo dostopati, zato se poslužujemo njegovih približkov.  

Video posnetek je sestavljen iz sekvence slik, to pa lahko opišemo kot funkcijo osvetljenosti slikovnega elementa $I(\vec{x},t)$, na poziciji $\vec{x} = [x~y]^\top$ ob času $t$ \cite{wedel2011stereo}. Gibanje oseb opazimo kot premikanje pikslov skozi čas, pri čemer predpostavimo, da osvetljenost posameznega piklsa ostaja konstantnta \cite{trucco1998introductory}. Stacionarnost osvetljenosti slikovnega elementa lahko opišemo z enačbo  

\begin{equation}
	\frac{d I(\vec{x}, t)}{dt} = \frac{\partial I}{\partial x} \frac{dx}{dt} + \frac{\partial I}{\partial y} \frac{dy}{dt} + \frac{\partial I}{\partial t} = 0,
\end{equation}

to pa lahko zapišemo z vektorjem hitrosti slikovnega elementa $\vec{u}$ v kompaktnejšo obliko

\begin{equation}\label{eq:opticni-tok}
	(\nabla I)^\top \vec{u} + I_t = 0.
\end{equation}

Enačba \eqref{eq:opticni-tok} predstavlja omejitev \textbf{optičnega toka} \cite{trucco1998introductory}. Če v enačbi \eqref{eq:opticni-tok} normaliziramo prostorski gradient $(\nabla I)$, v enačbi \eqref{eq:aperture-problem} opazimo, da lahko  določimo le hitrost, ki je vzporedna prostorskemu gradientu. Pojav je znan kot problem reže (angl. Aperture problem) \cite{trucco1998introductory}. 

\begin{equation}\label{eq:aperture-problem}
	\frac{(\nabla I)^\top \vec{u}}{\| \nabla I \|} = - \frac{E_t}{\| \nabla I \|} = u_n
\end{equation}

Problem reže si lahko razlagamo na način opazovanja gibanja črne daljice na beli podlagi skozi režo tako, da ne vidimo koncev. Zaradi omejene vizualne informacije lahko določimo hitrost le v pravokotni smeri na daljico \cite{trucco1998introductory}. 

Kadar imamo na slikovni ravnini več premikajočih slikovnih elementov, vektorsko polje hitrosti $\vec{u}$ imenujemo \textbf{optični tok} (angl. Optical flow) \cite{trucco1998introductory}. Optični tok je dobra aproksimacija polja gibanja v točkah visokega prostorskega gradienta svetlosti in konstantne osvetlitve.



\subsection{Metode estimacije optičnega toka}

Metode estimacije optičnega toka v grobem delimo na diferencialne in {ujemalne} metode \cite{trucco1998introductory}. Z \textbf{diferencialnimi metodami} računamo optični tok z uporabo parcialnih diferencialnih enačb ali minimizacijskimi metodami. Z metodami dobimo \textbf{gost optični kot}, kar pomeni, da je optični tok določen za vsak slikovni element \cite{trucco1998introductory}. Te metode zelo natačno opisujejo optični tok in ne proizvajajo vrednosti, ki lokalno odstopajo, zato je optični tok gladek \cite{brox2011large}.  Glavni problem teh metod je, da so računsko zelo zahtevne \cite{trucco1998introductory}.

Pri \textbf{ujemalnih metodah} računamo optični tok le na značilnih točkah \cite{trucco1998introductory}. Zaradi uporabe značilk so te metode lahko bolj efektivne, saj ne potrebujemo določevanja korespondenc za vse piksle. Prav tako se lahko uporabijo za računanje optičnega toka v realnem času, saj niso računsko zahtevne. Po \cite{trucco1998introductory} je največja težava teh metod, da računajo \textbf{redek optični tok}, saj je ta določen le za slikovne elemete, ki predstavljajo značilne točke. Prav tako delujejo dobro le pri majhnih premikih, ker temeljijo na Taylorjevi aproksimaciji enačbe \eqref{eq:opticni-tok} \cite{wedel2011stereo}. 

Na podlagi zgoraj opisanih lastnosti smo se odločili, da bomo v tem delu uporabili diferencialno metodo. Kljub višji računski zahtevnosti, ki ob današnji tehnologiji ne predstavlja več takega problema, smo želeli računati gost optični tok. Z gostim optičnim tokom tako dobimo natačno aproksimacijo polja gibanja za celotno telo. Prav tako nimamo problemov pri estimaciji energijske porabe za hitre gibe, kot bi bilo to v primeru uporabe ujemlanih metod. Ker je glavni namen uporaba in ne implementacija diferencialne metode, smo se osredotočili na Farneb{\"a}ck algoritem, ki je dostopen v knjižnici OpenCV.

\paragraph{Farneb{\"a}ck algoritem.}
Alogritem temelji na estimaciji premika z razčlenjevanjem polinoma  po enačbi \eqref{eq:polinom}, kjer je $\vec{A}$ simetrična matrika, $\vec{b}$ vektor in $c$ skalar \cite{farneback2003two}.

\begin{equation}\label{eq:polinom}
	f(\vec{x}) \sim \vec{x}^\top \vec{A} \vec{x} + \vec{b}^\top \vec{x} + c
\end{equation}

Ideja temelji na tem, da aproksimiramo okolico piksla s kvadratičnim polinomom, pri čemer želimo najti premik piksla na poziciji $\vec{x}$ z minimizacijo enačbe \eqref{eq:polinom-min} in omejitvijo \eqref{eq:omejitev-polinoma}. $\vec{A}_1(\vec{x})$ in $\vec{b}_1(\vec{x})$ sta razčlenitvena koeficienta za prvo sliko, $\vec{A}_2(\vec{x})$ in $\vec{b}_2(\vec{x})$ koeficienta za drugo sliko in $w(\Delta\vec{x})$ je utežna funkcija za sosedne točke.

\begin{align}
\vec{A}(\vec{x}) = & \frac{\vec{A}_1(\vec{x} + \vec{A}_2(\vec{x}))}{2} \\
\Delta\vec{b}(\vec{x}) = & - \frac{1}{2}\left(\vec{b}_2(\vec{x}) - \vec{b}_1(\vec{x})\right) 
\end{align}

\begin{equation}\label{eq:polinom-min}
\sum_{\Delta x \in I} w(\Delta\vec{x}) \| \vec{A}(\vec{x} + \Delta\vec{x})\vec{d}(\vec{x}) - \Delta\vec{b}(\vec{x} +\Delta\vec{x}) \|^2
\end{equation}

\begin{equation}\label{eq:omejitev-polinoma}
\vec{A}(\vec{x})\vec{d}(\vec{x}) = \Delta\vec{b}(\vec{x})
\end{equation}

Rešitev minimizacije enačbe \eqref{eq:polinom-min} je enačba \eqref{eq:polinom-resitev}

\begin{equation}\label{eq:polinom-resitev}
 \vec{d}(\vec{x}) = \left( \sum w \vec{A}^\top \vec{a} \right)^{-1} \sum w \vec{A}^\top \Delta\vec{b}
\end{equation}

Evaluacija algoritma je bila narejena v \cite{Geiger2012CVPR}. Rezultati so povzeti v tabeli \ref{tab:farneback}. Algoritem so preverjali s procesorjem z 1 jedrom \@ \SI{2.5}{GHz}.

\begin{table}
	\centering
    \begin{tabular}{S[table-format=2.2] S[table-format=2.2] S[table-format=2.1] S[table-format=2.1] S[table-format=3.2] S[table-format=1]}
    \toprule
    \multicolumn{1}{c}{\textbf{Out-Noc}} & \multicolumn{1}{c}{\textbf{Out-All}} & \multicolumn{1}{c}{\textbf{Avg-Noc}} & \multicolumn{1}{c}{\textbf{Avg-All}} & \multicolumn{1}{c}{\textbf{Gostota}} & \multicolumn{1}{c}{\textbf{Čas izvajanja}} \\
    \midrule
    47.59~\% & 54.00~\% & 17.3~px & 25.3~px & 100.00~\% & 1~s\\
    \bottomrule
    \end{tabular}
    \caption[Evaluacija Farneb{\"a}ck algoritma v KITTI Vision Benchmark 2012]{Evaluacija Farneb{\"a}ck algoritma v KITTI Vision Benchmark 2012 \cite{Geiger2012CVPR}. Metrika Out-Noc predstavlja procent pikslov, ki težijo k napakam v območju, kjer ni prekrivnosti. Out-all je procent pikslov, ki težijo k napakam v celoti. Avg-Noc je povprečna napaka disparitete v območjih neprekrivnsoti. Avg-All je povprečna napaka disparitete v celoti. Gostota predstavlja procent pikslov, za katere je metoda določila referenco \cite{Geiger2012CVPR}.}
    \label{tab:farneback}
\end{table}




\section{Prostorski tok}
% Teorija prostorskega toka
% Katero metodo smo mi uporabili in zakaj


\section{Deskriptorji}
% Težave optičnega toka in prostorskega toka.
% Kako bi te težave rešili z deskriptorji
% Našli specifične deskriptorje
Žal so klasične metode optičnega toka občutljive na šum, diskontinuitete gibanja ter spremembe v osvetljenosti objekta~\cite{brox2011large}, pri novejših pa še vedno obstaja problem pravilne ocene amplitude gibanja zaradi pojava paralakse~\cite{xu2012scale} -- objekti, ki so bolj oddaljeni od kamere imajo manjšo jakost optičnega toka. Zato smo se odločili, da gibanje iz video posnetkov predstavimo s histogrami orientiranega optičnega toka (HOOF) \cite{chaudhry2009histograms}, ki nekoliko izboljšajo robustnost.


\subsection{HOOF}
% Teorija teh deskriptorjev
% Kako smo jih mi uporabili
% Težave teh deskriptorjev
% Dodali nove deskriptorje

\subsection{HAAF}
% Opis dodatnih deskriptorjev
% Kako uporabili nove deskriptorje

\subsection{kalibracija velikosti glede na diagonalo}
% Zakaj naj bi bila ta rešitev dobra
% teorija tovrstne kalibracije
% Kako s to kalibracijo nismo popravili stvari
% In da smo ugotovili da obstaja še prostorski tok, ki bi nam rešil težave

\section{Metode manipulacije video posnetkov}
% Za testiranje optičnega toka in deskriptorjev, testiranje njihove robustnosti smo naredili več manipulacij, za boljšo predstavo o delovanju

% Za vse velja:
% Zakaj ta manipulacija
% Opis manipulacije
\subsection{Skaliranje slike}
\subsection{Projektivna transformacija}
\subsection{Združevanje posnetkov}
\subsection{Obrezovanje posnetkov}
\subsection{Simulacija vibracij kamere}


\section{Sledilnik}
% Zakaj smo uporabili sledilnik
% Na katere sledilnike smo ciljali 
% Primerjava sledilnikov
% Naši eksperimenti kateri je boljši
% 

\section{Matematični modeli}
\subsection{Jedra}
\subsection{rbf}
\subsection{ghi}
\subsection{linear}
\subsection{rbf-nu}
\subsection{Horizontalno skaliranje}
\subsection{Vertikalno skaliranje}
\subsection{Mrežno iskanje parametrov}

\section{Evaluacijske metrike}


\section{Materiali}
\subsection{Zlati standard}
\subsection{Senzorji v lab}
\subsection{Senzorji na terenu}
\subsection{Sinhornizacija časa}


\section{Združevanje slik iz dveh kinektov}