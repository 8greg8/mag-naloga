\section{Optični tok} \label{sec:opticni-tok}
% Teorija optičnega toka
% vrste optičnega toka
% Kateri tip smo mi uporabili
% Zakaj smo tega uporabili
% Kako smo ga uporabili
% Problemi naše predpostavke
% Rešitev s kalibracijo velikosti
Za namen razlage upoštevamo perpektivni model kamere, kjer je \cite{trucco1998introductory}:

\begin{itemize}
\item \textbf{optična os} enaka $Z$ osi koordinatnega sistema kamere.
\item \textbf{Gibajoči masni delec} je predstavljen enako kot v poglavju \ref{sec:model-gibanja}.
\item \textbf{Osvetlitev} se ne spreminja.
\end{itemize}

Sliko delca $\vec{q} = [x~y]^\top$ na slikovni ravnini $\mathit{\Omega} \subset \mathbb{R}^2$, kjer sta $x$ in $y$ slikovni koordinati, lahko predstavimo z enačbo

\begin{equation}\label{eq:slika-delca}
	\vec{q} = f \frac{\vec{p}}{Z},
\end{equation}

kjer je $f$ goriščna razdalja kamere \cite{trucco1998introductory}. Delec in njegova slika sta predstavljena na sliki \ref{fig:optical-flow} S časovnim odvodom enačbe \eqref{eq:slika-delca}, dobimo hitrost delca na slikovni ravnini:

\begin{equation}\label{eq:hitrost-slike-delca}
	\vec{u} = f \frac{Z\vec{v}-v_Z\vec{p}}{Z^2},
\end{equation}

kjer je $\vec{u} \in \mathcal{U} \subset \mathbb{R}^2$.




\begin{figure}
\centering
\begin{tikzpicture}%[tdplot_main_coords, scale=0.5]
[x={(0.8cm,0.4cm)}, y={(0cm,1cm)}, z={(0.8cm,-0.4cm)}, scale=0.5]

	% Coordinate system
    \coordinate (O) at (0,0,0);
    \coordinate (oo) at (0,0,5);
    \coordinate (y) at (0,5,0);
    \coordinate (z) at (0,0,15);
    \coordinate (x) at (10,0,0);
    \draw [axis] (O) -- (y) node [above] {$y$};
    \draw [base-axis] (O) -- (oo);
    \draw [axis] (O) -- (x) node [below] {$x$};
    \node (izhodisce) [below] at (O) {$o$};
    
    % Image plane
    \coordinate (ol) at (-5,0,5);
    \coordinate (or) at (5,0,5);
    \coordinate (ot) at (0,3,5);
    \coordinate (ob) at (0,-3,5);
    \coordinate (lb) at (-5,-3,5);
    \coordinate (rb) at (5,-3,5);
    \coordinate (lt) at (-5,3,5);
    \coordinate (rt) at (5,3,5);
    \draw [plane] (lb) -- (lt) -- (rt) -- (rb) -- cycle;
    \draw [dash] (ol) -- (or);
    \draw [dash] (ob) -- (ot);
    \node [xshift=3mm, yshift=5mm] at (lb) {$\mathit{\Omega}$};
    % Draw the rest of axis
    \draw [axis] (oo) -- (z) node [below] {$z$};
    
    % focal length
    \coordinate (of) at (-5,0,0);
   	\draw [dash] (O) -- (of);
    \draw [<->] (of) -- (ol) node [below] at (-5,0,2.5) {$f$};
    
    % delec
    \coordinate (p) at (5,3,10);
    \draw [fill=black] (p) circle (1.5mm) node [below] {$\vec{p}$};
    \draw [dash, name path=line 1] (O) -- (p);
    \draw [dash] (5,0,0) node [above] {$X$} -- (5,0,10);
    \draw [dash] (0,0,10) node [below] {$Z$} -- (5,0,10);
    \draw [dash] (5,0,10) -- (p);
    
    % hitrost
    \coordinate (v) at (6,4,10);
    \draw [velocity] (p) -- (v) node [above] {$\vec{v}$};
    \draw [dash] (O) -- (v);
    
    % slika delca
    \coordinate (q) at (2.5,1.5,5);
    \draw [fill=black] (q) circle (1mm) node [below] {$\vec{q}$};
    \draw [dash] (2.5,0,5) node [below] {$x$} -- (q);
    \draw [dash] (0,1.5,5) node [left] {$y$} -- (q);
    
    %hitrost
    \draw [velocity] (q) -- (3,2,5) node [above] {$\vec{u}$};

\end{tikzpicture}
\caption[Preslikava hitrosti delca na slikovno ravnino $\mathit{\Omega}$]{ Preslikava hitrosti delca na slikovno ravnino $\mathit{\Omega}$. Gibajoči delec $\vec{p}$ ima sliko $\vec{q}$. Hitrost delca $\vec{v}$ ima sliko hitrosti $\vec{u}$, ki predstavlja idealni vektor gibanja. Koordinatni sistem predstavlja sistem kamere.}
\label{fig:optical-flow}
\end{figure}




Razširjena oblika enačbe \eqref{eq:hitrost-slike-delca}, kjer upoštevamo \eqref{eq:gibanje}, je zapisana z enačbo \eqref{eq:hitrost-slike-delca-raz} \cite{trucco1998introductory}. Prvi člen v posamezni enačbi predstavlja \textbf{translatorni del}, ostali členi pa sodijo v \textbf{rotacijski del}.

\begin{equation}\label{eq:hitrost-slike-delca-raz}
\begin{aligned}
	u_x = & \frac{T_Z x - T_X f}{Z} - \omega_Y f + \omega_Z y + \frac{\omega_X x y}{f} - \frac{\omega_Y x^2}{f} \\
    u_y = & \frac{T_Z y - T_Y f}{Z} - \omega_X f + \omega_Z x + \frac{\omega_Y x y}{f} - \frac{\omega_X y^2}{f}
\end{aligned}
\end{equation}

Kadar imamo na slikovni ravnini več slik delcev, množico vektorjev hitrosti $\vec{u}$ imenujemo \textbf{polje gibanja} (angl. Motion Field) $\vec{G} : \mathit{\Omega} \to \mathcal{U}$, kjer velja $ \vec{q} \mapsto \vec{u}$ \cite{trucco1998introductory}. Polje gibanja $\vec{G}$ lahko razumemo kot projekcijo polja hitrosti $\vec{H}$ na slikovno ravnino, zato ta predstavlja idealno rekonstrukcijo gibanja. V praksi do polja gibanja ne moremo dostopati, zato se poslužujemo njegovih približkov.  

Video posnetek je sestavljen iz sekvence slik, to pa lahko opišemo kot funkcijo osvetljenosti slikovnega elementa $I(\vec{x},t)$, na poziciji $\vec{x} = [x~y]^\top$ ob času $t$ \cite{wedel2011stereo}. Gibanje oseb opazimo kot premikanje pikslov skozi čas, pri čemer predpostavimo, da osvetljenost posameznega piklsa ostaja konstantnta \cite{trucco1998introductory}. Stacionarnost osvetljenosti slikovnega elementa lahko opišemo z enačbo  

\begin{equation}
	\frac{d I(\vec{x}, t)}{dt} = \frac{\partial I}{\partial x} \frac{dx}{dt} + \frac{\partial I}{\partial y} \frac{dy}{dt} + \frac{\partial I}{\partial t} = 0,
\end{equation}

to pa lahko zapišemo z vektorjem hitrosti slikovnega elementa $\vec{w} \in \mathcal{W} \subset \mathbb{R}^2$ v kompaktnejšo obliko

\begin{equation}\label{eq:opticni-tok}
	(\nabla I)^\top \vec{w} + I_t = 0.
\end{equation}

Enačba \eqref{eq:opticni-tok} predstavlja \textbf{omejitev optičnega toka} \cite{trucco1998introductory}. Če v enačbi \eqref{eq:opticni-tok} normaliziramo prostorski gradient $(\nabla I)$, v enačbi \eqref{eq:aperture-problem} opazimo, da lahko  določimo le hitrost, ki je vzporedna prostorskemu gradientu. Pojav je znan kot problem reže (angl. Aperture problem) \cite{trucco1998introductory}. 

\begin{equation}\label{eq:aperture-problem}
	\frac{(\nabla I)^\top \vec{w}}{\| \nabla I \|} = - \frac{I_t}{\| \nabla I \|} = w_n
\end{equation}

\textbf{Problem reže} si lahko razlagamo na način opazovanja gibanja daljice na beli podlagi skozi režo tako, da ne vidimo koncev. Zaradi omejene vizualne informacije lahko določimo hitrost le v pravokotni smeri na daljico \cite{trucco1998introductory}. Razlaga je predstavljena na sliki \ref{fig:aperture-problem}.




\begin{figure}[htb]
\centering
\begin{tikzpicture}[scale=0.7]
\tikzset{aperture/.style = {
fill=teal!50, 
draw=teal!50!black!80
}}
\tikzset{stick/.style = {
fill=orange!50!black!50, 
draw=orange!50!black!80, solid, thick,
minimum width = 1mm, minimum height=7cm
}}
  \begin{scope}
  		\coordinate (top) at (10,10);
        \coordinate (bottom) at (0,0);
       	\coordinate (center) at (5,5);

  		% palica
        \coordinate (pc) at (4,4);
        \path (pc) node[stick,rotate = 45]{};
        \draw [fill] (pc) circle (0.5mm);
        \coordinate (pc2) at (6,4);
        \path (pc2) node[stick, rotate = 45, dashed, fill=none]{};
        \draw [fill] (pc2) circle (0.5mm);
        
        % vektor
        \draw [velocity, draw=black] (pc) -- ++(1,1) node [below] {${\vec{w}_n}$};
        \draw [velocity, draw=black] (pc) -- (pc2) node [below] {$\vec{w}$};
  		% aperture	
		\draw [aperture, opacity=0.8] (bottom) rectangle (top) (center) circle (20mm);  
  \end{scope}


\end{tikzpicture}
\caption[Problem reže]{Problem reže. Ker skozi režo ne vidim koncev daljice, lahko določimo le hitrost v pravokotni smeri na daljico \cite{trucco1998introductory}.}
\label{fig:aperture-problem}
\end{figure}




Kadar imamo na slikovni ravnini več premikajočih slikovnih elementov, vektorsko polje hitrosti $\vec{w}$ imenujemo \textbf{optični tok} (angl. Optical flow) $\vec{O}: \mathit{\Omega} \to \mathcal{W}$, kjer velja $ \vec{q} \mapsto \vec{w}$ \cite{trucco1998introductory}. Optični tok je dobra aproksimacija polja gibanja v točkah visokega prostorskega gradienta svetlosti in konstantne osvetlitve.



\subsection{Metode estimacije optičnega toka}

Metode estimacije optičnega toka $\vec{O}$ v grobem delimo na diferencialne in {ujemalne} metode \cite{trucco1998introductory}. Z \textbf{diferencialnimi metodami} računamo optični tok z uporabo parcialnih diferencialnih enačb ali minimizacijskimi metodami. Z metodami dobimo \textbf{gost optični tok}, kar pomeni, da je optični tok določen za vsak slikovni element \cite{trucco1998introductory}. Te metode zelo natačno opisujejo optični tok in ne proizvajajo vrednosti, ki lokalno odstopajo, zato je optični tok gladek \cite{brox2011large}.  Glavni problem teh metod je, da so računsko zelo zahtevne \cite{trucco1998introductory}.

Pri \textbf{ujemalnih metodah} računamo optični tok le na značilnih točkah \cite{trucco1998introductory}. Zaradi uporabe značilk so te metode lahko bolj efektivne, saj ne potrebujemo določevanja korespondenc za vse piksle. Prav tako se lahko uporabijo za računanje optičnega toka v realnem času, saj niso računsko zahtevne. Po \cite{trucco1998introductory} je največja težava teh metod, da računajo \textbf{redek optični tok}, saj je ta določen le za slikovne elemete, ki predstavljajo značilne točke. Prav tako delujejo dobro le pri majhnih premikih, ker temeljijo na Taylorjevi aproksimaciji enačbe \eqref{eq:opticni-tok} \cite{wedel2011stereo}. 

Na podlagi zgoraj opisanih lastnosti smo se odločili, da bomo v tem delu uporabili diferencialno metodo. Kljub višji računski zahtevnosti, ki ob današnji tehnologiji ne predstavlja več takega problema, smo želeli računati gost optični tok. Z gostim optičnim tokom tako dobimo natačno aproksimacijo polja gibanja za celotno telo. Prav tako nimamo problemov pri estimaciji energijske porabe za hitre gibe, kot bi bilo to v primeru uporabe ujemlanih metod. Ker je glavni namen uporaba in ne implementacija diferencialne metode, smo se osredotočili na Farneb{\"a}ck algoritem, ki je dostopen v knjižnici OpenCV.

\paragraph{Farneb{\"a}ck algoritem.}
Alogritem temelji na estimaciji premika z razčlenjevanjem polinoma  po enačbi \eqref{eq:polinom}, kjer je $\vec{A}$ simetrična matrika, $\vec{b}$ vektor in $c$ skalar \cite{farneback2003two}.

\begin{equation}\label{eq:polinom}
	f(\vec{x}) \sim \vec{x}^\top \vec{A} \vec{x} + \vec{b}^\top \vec{x} + c
\end{equation}

Ideja temelji na tem, da aproksimiramo okolico piksla s kvadratičnim polinomom, pri čemer želimo najti premik piksla na poziciji $\vec{x}$ z minimizacijo enačbe \eqref{eq:polinom-min} in omejitvijo \eqref{eq:omejitev-polinoma}. $\vec{A}_1(\vec{x})$ in $\vec{b}_1(\vec{x})$ sta razčlenitvena koeficienta za prvo sliko, $\vec{A}_2(\vec{x})$ in $\vec{b}_2(\vec{x})$ koeficienta za drugo sliko in $w(\Delta\vec{x})$ je utežna funkcija za sosedne točke.

\begin{align}
\vec{A}(\vec{x}) = & \frac{\vec{A}_1(\vec{x} + \vec{A}_2(\vec{x}))}{2} \\
\Delta\vec{b}(\vec{x}) = & - \frac{1}{2}\left(\vec{b}_2(\vec{x}) - \vec{b}_1(\vec{x})\right) 
\end{align}

\begin{equation}\label{eq:polinom-min}
\sum_{\Delta x \in I} w(\Delta\vec{x}) \| \vec{A}(\vec{x} + \Delta\vec{x})\vec{d}(\vec{x}) - \Delta\vec{b}(\vec{x} +\Delta\vec{x}) \|^2
\end{equation}

\begin{equation}\label{eq:omejitev-polinoma}
\vec{A}(\vec{x})\vec{d}(\vec{x}) = \Delta\vec{b}(\vec{x})
\end{equation}

Rešitev minimizacije enačbe \eqref{eq:polinom-min} je enačba \eqref{eq:polinom-resitev}

\begin{equation}\label{eq:polinom-resitev}
 \vec{d}(\vec{x}) = \left( \sum w \vec{A}^\top \vec{a} \right)^{-1} \sum w \vec{A}^\top \Delta\vec{b}
\end{equation}

Evaluacija algoritma je bila narejena v \cite{Geiger2012CVPR}. Rezultati so povzeti v tabeli \ref{tab:farneback}. Algoritem so preverjali s procesorjem z 1 jedrom \@ \SI{2.5}{GHz}.

\begin{table}
	\centering
    \begin{tabular}{S[table-format=2.2] S[table-format=2.2] S[table-format=2.1] S[table-format=2.1] S[table-format=3.2] S[table-format=1]}
    \toprule
    \multicolumn{1}{c}{\textbf{Out-Noc}} & \multicolumn{1}{c}{\textbf{Out-All}} & \multicolumn{1}{c}{\textbf{Avg-Noc}} & \multicolumn{1}{c}{\textbf{Avg-All}} & \multicolumn{1}{c}{\textbf{Gostota}} & \multicolumn{1}{c}{\textbf{Čas izvajanja}} \\
    \midrule
    47.59~\% & 54.00~\% & 17.3~px & 25.3~px & 100.00~\% & 1~s\\
    \bottomrule
    \end{tabular}
    \caption[Evaluacija Farneb{\"a}ck algoritma v KITTI Vision Benchmark 2012]{Evaluacija Farneb{\"a}ck algoritma v KITTI Vision Benchmark 2012 \cite{Geiger2012CVPR}. Metrika Out-Noc predstavlja procent pikslov, ki težijo k napakam v območju, kjer ni prekrivnosti. Out-all je procent pikslov, ki težijo k napakam v celoti. Avg-Noc je povprečna napaka disparitete v območjih neprekrivnsoti. Avg-All je povprečna napaka disparitete v celoti. Gostota predstavlja procent pikslov, za katere je metoda določila referenco \cite{Geiger2012CVPR}.}
    \label{tab:farneback}
\end{table}




\section{Prostorski tok}
% Teorija prostorskega toka
% Katero metodo smo mi uporabili in zakaj
Optični tok $\vec{O}$ predstavlja aproksimacijo polja gibanja $\vec{G}$, ta pa je projekcija polja hitrosti $\vec{H}$ na slikovno ravnino $\mathit{\Omega}$ \cite{trucco1998introductory}. Če pogledamo z druge perspektive, ni optični tok $\vec{O}$ nič drugega kot projekcija aproksimacije polja hitrosti $\vec{H}$, ki jo po analogiji lahko imenujemo \textbf{prostorski tok} (angl. Scene Flow) \cite{vedula1999three}. 

Za namen razlage upoštevamo enake omejitve kamere, masnega delca in osvetlitve, kot v poglavju \ref{sec:opticni-tok}. Predpostavimo, da imamo v prostoru površino $f(x,y,z) = 0$ na kateri imamo gibajoč točkovni delec $\vec{p} = \vec{p}(t)$. Na slikovni ravnini $\mathit{\Omega}$ imamo njegovo sliko $\vec{q}$ \cite{vedula1999three}. Vizualni prikaz prizora lahko vidimo na sliki \ref{fig:scene-flow}. 




\begin{figure}
\centering
\begin{tikzpicture}%[tdplot_main_coords, scale=0.5]
[x={(0.8cm,0.4cm)}, y={(0cm,1cm)}, z={(0.8cm,-0.4cm)}, scale=0.5]

	% Coordinate system
    \coordinate (O) at (0,0,0);
    \coordinate (oo) at (0,0,5);
    \coordinate (y) at (0,5,0);
    \coordinate (z) at (0,0,15);
    \coordinate (x) at (10,0,0);
    \draw [axis] (O) -- (y) node [above] {$y$};
    \draw [base-axis] (O) -- (oo);
    \draw [axis] (O) -- (x) node [below] {$x$};
    \node (izhodisce) [below] at (O) {$o$};
    
    % Image plane
    \coordinate (ol) at (-5,0,5);
    \coordinate (or) at (5,0,5);
    \coordinate (ot) at (0,3,5);
    \coordinate (ob) at (0,-3,5);
    \coordinate (lb) at (-5,-3,5);
    \coordinate (rb) at (5,-3,5);
    \coordinate (lt) at (-5,3,5);
    \coordinate (rt) at (5,3,5);
    \draw [plane] (lb) -- (lt) -- (rt) -- (rb) -- cycle;
    \draw [dash] (ol) -- (or);
    \draw [dash] (ob) -- (ot);
    \node [xshift=3mm, yshift=5mm] at (lb) {$\mathit{\Omega}$};
    % Draw the rest of axis
    \draw [axis] (oo) -- (z) node [below] {$z$};
    
    % focal length
    \coordinate (of) at (-5,0,0);
   	\draw [dash] (O) -- (of);
    \draw [<->] (of) -- (ol) node [below] at (-5,0,2.5) {$f$};
    
    % povrsina
    \begin{scope}
    \coordinate (p0) at (3,1,12);
    \coordinate (p1) at (3,5,12);
    \coordinate (p2) at (7,5,12);
    \coordinate (p3) at (7,1,12);
   
    \draw [plane, fill=purple!20, draw=purple!50!black!50,] 
    			   (p0) .. controls (3,4,10) .. (p2)
                        .. controls (7,3,10) .. (p3) 
                        .. controls (5,1,10) .. cycle;
    \node at (6,2,12) {$f$};
    \end{scope}
    
    
    
    % delec
    \coordinate (p) at (5,3,10);
    \draw [fill=black] (p) circle (1.5mm) node [below] {$\vec{p}$};
    \draw [dash, name path=line 1] (O) -- (p);
    %\draw [dash] (5,0,0) node [above] {$X$} -- (5,0,10);
    %\draw [dash] (0,0,10) node [below] {$Z$} -- (5,0,10);
    %\draw [dash] (5,0,10) -- (p);
    
    % hitrost
    \coordinate (v) at (6,4,10);
    \draw [velocity] (p) -- (v) node [above] {$\vec{\mu}$};
    \draw [dash] (O) -- (v);
    
    % slika delca
    \coordinate (q) at (2.5,1.5,5);
    \draw [fill=black] (q) circle (1mm) node [below] {$\vec{q}$};
    \draw [dash] (2.5,0,5) node [below] {$x$} -- (q);
    \draw [dash] (0,1.5,5) node [left] {$y$} -- (q);
    
    %hitrost
    \draw [velocity] (q) -- (3,2,5) node [above] {$\vec{w}$};
\end{tikzpicture}
\caption[Vizualni prikaz vektorja prostorskega toka $\vec{w}$]{Vizualni prikaz vektorja prostorskega toka $\vec{w}$. V prostoru imamo površino $f$ na kateri leži gibajoči točkovni delec $\vec{p}$ \cite{vedula1999three}. Na slikovni ravnini $\mathit{\Omega}$ imamo sliko delca $\vec{q}$ s prostorskim tokom $\vec{w}$.}
\label{fig:scene-flow}
\end{figure}



Ker je slika projekcija delca na slikovno ravnino $\mathit{\Omega}$, lahko zapišemo $\vec{q} = \vec{q}(\vec{p})$. Hitrost slike določimo po enačbi \eqref{eq:opticni-tok-sf}, ki predstavlja enačbo vektorja optičnega toka $\vec{w}$, kot projekcijo prostorskega toka \cite{vedula1999three}. 


\begin{equation}\label{eq:opticni-tok-sf}
	\vec{w} = \frac{d\vec{q}}{dt} = \frac{\partial \vec{q}}{\partial \vec{p}}\frac{d\vec{p}}{dt}
\end{equation}

Če predpostavimo, da imamo dovoj informacije o sistemu, da lahko določimo inverzno funkcijo $\vec{p} = \vec{p}(\vec{q},t)$, kjer je masni delec $\vec{p}$ projekcija slike $\vec{q}$, lahko določimo njegovo hitrost $\vec{\mu} \in \mathcal{\Mu} \subset \mathbb{R}^3$ z enačbo \eqref{eq:scene-flow}. Slednja je sestavljena iz dveh delov. Prvi člen je projekcija vektorja optičnega toka $\vec{w}$ na tangentno ravnino površine $f$, v točki, kjer se nahaja delec $\vec{p}$ \cite{vedula1999three}. Drugi člen je hitrost spreminjanja oddaljenosti delca od slikovne ravnine $\mathit{\Omega}$, ko slika delca stoji na miru. 

\begin{equation}\label{eq:scene-flow}
	\vec{\mu} = \frac{d\vec{p}}{dt} = \frac{\partial \vec{p}}{\partial \vec{q}} \frac{d\vec{q}}{dt} + \left.\frac{\partial \vec{p}}{\partial t}\right|_\vec{q}
\end{equation}

Kadar imamo v prostoru več, med seboj neodvisnih premikajočih masnih delcev, vektorsko polje hitrosti $\vec{\mu}$ imenujemo prostorski tok (angl. Scene Flow) $\vec{S}: \mathcal{W} \times \mathbb{R} \to \mathcal{\Mu}$, kjer velja $(\vec{w}, \dot{Z}) \mapsto \vec{\mu}$ \cite{yan2016scene}. $\dot{Z}$ predstavlja hitrost spreminjanja globine.

\subsection{Metode estimacije prostorskega toka}
Konvencionalne metode estimacije prostorskega toka so se razvile iz optičnega toka, in dodatne informacije o globini \cite{yan2016scene}. Slednjo lahko pridobimo s parom stereo kamer ali z uporabo sistemov večih kamer \cite{jaimez2015primal}. Vektor hitrosti prostorskega toka lahko v takih sistemih aproksimiramo z $\vec{\mu} = \left[w_x~w_y~\dot{d}\right]^\top$, kjer sta $(w_x, w_y)$ komponenti vektorja optičnega toka $\vec{w}$, $\dot{d}$ pa časovna sprememba disparitete \cite{yan2016scene}.

Z razvojem RGB-D kamer, kjer RGB predstavlja barvno sliko, D pa globinsko sliko, smo dobili cenovno dostopne in natančne sisteme, ki omogočajo implementacijo hitrih algoritmov prostorskega toka \cite{yan2016scene,jaimez2015primal}. Ti večinoma temeljijo na globalni variacijski metodi, kjer rešujemo minimizacijski problem 

\begin{equation}\label{eq:minimizacijski-problem}
	\min_{\mu}\{E_D(\vec{\mu}) + E_R(\vec{\mu})\}
\end{equation}

V podatkovnem delu funkcionala \eqref{eq:minimizacijski-problem} $E_D(\vec{\mu})$ \eqref{eq:podatkovni-del} upoštevamo konstantno osvetljenost \eqref{eq:konstantna-osvetljenost} in konsistentnost spreminjana globine \eqref{eq:konsistentnost-globine}, kjer je  $\Psi$ cenilka. Ponavadi je uporabljena $L_2$ norma $\Psi(x) = \| x \|^2$ \cite{yan2016scene}. $\alpha$ v \eqref{eq:podatkovni-del} predstavlja utež.

\begin{equation}\label{eq:konstantna-osvetljenost}
	E_{KO} = \sum_\Omega \Psi( I(\vec{q} + \vec{w}) - I(\vec{q}))
\end{equation}

\begin{equation}\label{eq:konsistentnost-globine}
	E_{KG} = \sum_\Omega \Psi\left( Z(\vec{q} + \Delta \vec{q}) - Z(\vec{q}) - \dot{Z}(\vec{q}))\right)
\end{equation}

\begin{equation}\label{eq:podatkovni-del}
	E_D = E_{KO} + \alpha E_{KG}
\end{equation}

V regularizacijskem delu funkcionala \eqref{eq:minimizacijski-problem} $E_R(\vec{\mu})$ pa uporabimo enačbo \eqref{eq:regularizacijski-del}

\begin{equation}\label{eq:regularizacijski-del}
	E_R = \sum_\Omega \Psi\left( \nabla w_x \right) + \Psi\left( \nabla w_y \right) + \Psi\left( \nabla \dot{Z} \right)
\end{equation}

Ker smo v našem delu za računanje prostorskega toka uporabili Kinect senzorje, smo potrebovali metodo, ki temelji na RGB-D podatkih. Ker je glavni namen uporaba in ne implementacija algoritma prostorskega toka, smo se osredotočili na PD-Flow algoritem, ki je javno dostopen \cite{jaimez2015primal}.

\paragraph{PD-Flow algoritem.}
Algoritem spada pod globalne variacijske metode \cite{jaimez2015primal}. Za cenilko $\Psi$ v \eqref{eq:konstantna-osvetljenost} in \eqref{eq:konsistentnost-globine} avtorji uporabljajo $L_1$ normo. Za utež $\alpha$ v podatkovnem delu funkcionala \eqref{eq:podatkovni-del} se v tem algorimu uporablja funkcija 

\begin{equation}\label{eq:utez}
 \alpha(x,y) = \frac{\mu_0}{1 + k_\mu \left( \frac{\partial Z^2}{\partial x} + \frac{\partial Z^2}{\partial y} + \frac{\partial Z^2}{\partial t} \right)},
\end{equation}

kjer sta empirično določena parametra $\mu_0 = 75$ in $k_mu = 1000$. Za izračun podatkovnega dela uporabljajo hierarhično metodo grajenja slikovne piramide, pri tem pa uporabljajo linearizacijo podatkovnega dela \eqref{eq:podatkovni-del} \cite{jaimez2015primal}.

Jaimez et al. v delu \cite{jaimez2015primal} predstavi nov regularizacijski del \eqref{eq:regularizacijski-del-pdflow}, kjer upošteva še geometrijo prizora s faktorjem $\vec{r}$ \eqref{eq:faktor-prizora}. Z njim upošteva, da lahko sosednji slikovni elementi predstavljajo točke v prostoru, ki so si različno oddaljene.

\begin{equation}\label{eq:regularizacijski-del-pdflow}
E_R = \sum_\Omega \Psi\left( (\nabla w_x)^\top \vec{r} \right) + \Psi\left( (\nabla w_y)^\top \vec{r} \right) + \Psi\left( (\nabla \dot{Z})^\top \vec{r} \right)
\end{equation}

\begin{equation}\label{eq:faktor-prizora}
\vec{r} =
\begin{bmatrix}
\frac{1}{\sqrt{\frac{\partial X^2}{\partial x} + \frac{\partial Z^2}{\partial x}}} &
\frac{1}{\sqrt{\frac{\partial Y^2}{\partial y} + \frac{\partial Z^2}{\partial y}}}
\end{bmatrix}^\top
\end{equation}

Evaluacija PD-Flow algoritma je prikazana v tabeli \ref{tab:pdflow}. V tabeli so zapisani še rezultati RGB-D flow algoritma, ki za cenilko $\Psi$ ravno tako uporablja $L_1$ normo \cite{jaimez2015primal}. Opazimo lahko, da se PD-Flow po metrikah bolje odnese. Največji izboljšanje vidimo pri času izvajanja algoritma.

\begin{table}
	\centering
    \begin{tabular}{l S[table-format=1.3] S[table-format=2.3] S[table-format=3.3] S[table-format=3.3]}
    \toprule
    \textbf{Algoritem} & \multicolumn{1}{c}{\textbf{NRMS-V}} & \multicolumn{1}{c}{\textbf{AAE}} & \multicolumn{1}{c}{\textbf{Čas izvajanja [s]}} & \multicolumn{1}{c}{\textbf{MAX-V [m]}} \\
    \midrule
    \textbf{PD-Flow} & \boldentry{1.3}{0.068} & \boldentry{2.3}{6.653} & \boldentry{3.3}{7.150} & 0.111 \\
    RGB-D flow & 0.096 & 15.58 & 119.1 & 0.111 \\
    \bottomrule
    \end{tabular}
    \caption[Evaluacija PD-Flow algoritma]{Evaluacija PD-Flow algoritma in primerjava z algoritmom RGB-D flow, ki uporablja enako cenilko $\Psi$ \cite{jaimez2015primal}. Za metrike se uporabljata povprečna kotna napaka (AAE) in normaliziran koren srednje kvadratične napake magnitude hitrosti (NRMS-V), kjer se največja magnituda (MAX-V) uporablja za normalizacijo \cite{jaimez2015primal}. Opazimo lahko, da se PD-Flow po metrikah bolje odnese. Največji izboljšanje vidimo pri času izvajanja algoritma. Odebeljene vrednosti predstavljajo najboljšo vrednost.}
    \label{tab:pdflow}
\end{table}

