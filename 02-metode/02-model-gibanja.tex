\section{Model gibanja}\label{sec:model-gibanja}
% Ponovi teorijo energijske porabe in kako pridemo do gibanja
% Dokazi, da lahko gibanje opazujemo s kamerami
% Teorija, da je najboljši približek optični tok
Da mišice lahko s svojimi kontrakcijami spravijo telo v pogon, potrebujejo energijo, pri tem pa se del energije pretvarja v toploto, ki ji pravimo energijska poraba \cite{scott2005misconceptions}. Energijska poraba nastaja zaradi gibanja telesa, zato jo lahko določimo z opazovanjem kinematike \cite{levine2005measurement}. Zaradi omejitev kontaktnih senzorjev in slabih lastnosti srčnega utripa so v študijah, ki so opisane v poglavju \ref{sec:podobna-dela} dokazali, da lahko z opazovanjem kinematike z računalniškim vidom določimo energijsko porabo. 

Naj bo delec z maso $m$ predstavljen kot točka prizora $\vec{P} = [X~Y~Z]^\top$ , kjer $X$, $Y$ in $Z$ predstavljajo koordinate na posameznih oseh v koordinatnem sistemu kamere \cite{trucco1998introductory}. Gibanje delca $\vec{P}$ lahko predstavimo z vektorjem hitrosti $\vec{V} = [V_X~V_Y~V_Z]^\top$, kjer so $V_X$, $V_Y$ in $V_Z$ hitrosti glede na osi. Kadar imamo v prostoru več masnih delcev, množico vektorjev hitrosti imenujemo \textbf{polje hitrosti} (angl. Velocity Field) \cite{trucco1998introductory}. 

Relativno gibanje delca $P$ glede na koordinatno izhodišče kamere $O$ lahko opišemo kot:

\begin{equation}
	\vec{V} = -\vec{T}-\omega\times\vec{P},
\end{equation}

kjer je $\vec{T}$ translatorna hitrost in $\omega$ kotna hitrost \cite{trucco1998introductory}. Po komponentah lahko gibanje opišemo z enačbo \eqref{eq:gibanje}

\begin{equation} \label{eq:gibanje}
	\begin{bmatrix}
	V_x \\ V_Y \\ V_Z
	\end{bmatrix}
    =
    \begin{bmatrix}
    - T_X - \omega_Y Z + \omega_Z Y \\
    - T_Y - \omega_Z X + \omega_X Z \\
    - T_Z - \omega_X Y + \omega_Y X
    \end{bmatrix}.
\end{equation}

Gibanje telesa v prostoru torej lahko opišemo s poljem hitrosti gibanja.

%Pri najbolj splošni obliki detekcije gibanja iz video posnetkov, moramo izračunati gibanje posameznega slikovnega elementa \cite{szeliski2010computer}, tako imenovani \emph{optični tok}, ki pod določenimi predpostavkami predstavlja dober približek gibanja objekta na sliki. Žal so klasične metode optičnega toka občutljive na šum, diskontinuitete gibanja ter spremembe v osvetljenosti objekta~\cite{brox2011large}, pri novejših pa še vedno obstaja problem pravilne ocene amplitude gibanja zaradi pojava paralakse~\cite{xu2012scale} -- objekti, ki so bolj oddaljeni od kamere imajo manjšo jakost optičnega toka. Zato smo se odločili, da gibanje iz video posnetkov predstavimo s histogrami orientiranega optičnega toka (HOOF) \cite{chaudhry2009histograms}, ki nekoliko izboljšajo robustnost. Povezavo gibanja športnika z značilkami HOOF prikazuje slika, kot sledi.  

%V idealnem primeru lahko gibanje delca z maso $m$ razčlenimo na komponente hitrosti v vertikalni in horizontalni smeri. Gibanje v smeri $\vec{v}_{y+}$ predstavlja največji napor, saj mora mišica opravljati delo, ki kljubuje sili gravitacije. Nasprotno je lahko gibanje v smeri $\vec{v}_{y-}$ relativno nenaporno. Gibanje v smereh $\vec{v}_{x+}$ in $\vec{v}_{x-}$ je po vloženi energiji nekje vmes. Dela, ki pripada gibanju v poljubni smeri, ne znamo točno oceniti, s pomočjo značilk HOOF pa se ga lahko \emph{naučimo}.

