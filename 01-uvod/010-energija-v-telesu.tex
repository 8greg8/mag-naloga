\section{Energija v človeškem telesu}\label{sec:energija}
Človeško telo lahko energijo porabi na tri različne načine:

\begin{itemize}
\item bazalna metabolična stopnja,
\item termični efekt hrane in
\item energijska poraba zaradi fizične aktivnosti \cite{levine2005measurement}.
\end{itemize}

\textbf{Bazalna metabolična stopnja} je uporabljena energija v času mirovanja (zjutraj, ko se zbudimo). Za povprečnega človeka predstavlja okoli \SI{60}{\%} dnevne porabe energije \cite{levine2005measurement}. \textbf{Termični efekt hrane} predstavlja porabo energije, ki je povezana s prebavo in shranjevanjem hrane. Predstavlja okoli \SI{10}{\%} dnevne porabe energije \cite{levine2005measurement}. \textbf{Energijska poraba zaradi fizične aktivnosti} pa predstavlja energijo, ki jo trošijo mišice. 

Da mišice lahko s svojimi kontrakcijami spravijo telo v pogon, potrebujejo mišične celice energijo, ki je shranjena v obliki molekulskih vezi adenozintrifosfata (ATP) \cite{scott2005misconceptions}. Z razgradnjo ATP, celice dobijo potrebno energijo za kontrakcije, nato pa ponovno sintetizirajo ATP s pomočjo metabolizma iz amino kislin, ogljikovih hidratov in maščobnih kislin \cite{scott2005misconceptions,patel2017aerobic}. Ker sta ATP razgradnja in ponovna sinteza termodinačno irreversibilni reakciji, se del enrgije pretvarja v toploto, to pa imenujemo energijska poraba \cite{scott2005misconceptions}. ATP molekule tako predstavljajo energijsko kapaciteto, zato omejitev sinteze ATP dejansko povzroča omejitev energijske porabe in s tem zmogljivost sistema \cite{sahlin1998energy}. 

Za ponovno sintezo ATP molekul, lahko celice uporabljajo aerobni ali anaerobni metabolizem \cite{scott2005misconceptions}. Pri \textbf{aerobnem metabolizmu} celice uporabljajo kisik ($\mathrm{O}_2$) \cite{patel2017aerobic}. Ta se pojavlja večinoma pri ponavljajočih ritmičnih gibih in manj intezivni fizični aktivnosti, kot je kolesarjenje, daljši tek, ples, itd. \cite{patel2017aerobic}. Ker se aerobni procesi uporabljajo za dolgotrajnejše dejavnosti, imajo visoko kapaciteto in nizko moč \cite{sahlin1998energy}. Njihovo kapaciteto lahko zmerimo s pomočjo zmogljivosti kardio-respiratornega sistema \cite{patel2017aerobic}. Večja dobava kisika, bo posledično omogočila več aerobnega metabolizma in proizvajanja ATP molekul. Za kriterij aerobne kapacitete se je tako uveljavilo merjenje porabe kisika (${VO}_2$).

Anaerobni metabolizem se pojavi, ko primanjkuje kisika, za produkcijo ATP molekul \cite{patel2017aerobic}. Formacija ATP poteka preko glikolize, kar povzroča nizko raven ATP, sintezo mlečne kisline in akumulacijo laktata v krvi. Mlečna kislina zakisa mišice, to pa vodi v mišično utrujenost \cite{sahlin1998energy}. Anaerobni procesi se pojavijo ob intenzivnih aktivnostih, ki trajajo kratek čas (sprint, dviganje uteži...) \cite{patel2017aerobic}. Anaerobni procesi imajo tako veliko moč, vendar nizko kapaciteto \cite{sahlin1998energy}.